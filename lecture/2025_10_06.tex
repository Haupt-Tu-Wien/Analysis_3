\chapter{}
\section{Organisatorisches}
Alle Infomationen sind Online zu finden aber für die form von der Mitschrift sollte hier etwas stehen.:
Die LVa ist eine UE + Vo. 
\subsection{Übung}
Für die Übung benötigt man ca. 60\% der Käuze und zwei positive Tafeleistungen.
Man darf einmal fehlen + nachbringen aber ohne guten Grund. Ansonsten braucht man einen guten grund (meine Oma ist gestoreben,
ist ab dem zweiten mal kein guter Grund).
\subsection{Vorlesung}
Die Vorlesung ist ein Tafelvortag. 
Prüfung teilt sich auf in schriftlich und mündlich ... steht alles in der Email. für die Form reicht der text

\section{Topologische Grundbegriffe} 

Da wir für viele Themen die in Analysis 3 behandelt werden, Resultate aus der Mastheorie benötigen,
fangen wir mit einigen Grundlagen der Topologie an. Da man für Topologie nix braucht (man muss nur ein bisschen mit Mengen spielen).

\bigskip

Im Folgenden geht es darum den Konvergenz Begriff zu verallgemeinern. 
Dazu betrachten wir Räume, gerade noch soviel Struktur tragen, dass man von stetigen
Funktionen, Grenzwerten, Kompaktheit und so weiter sprechen kann.

Zur wiederholung betrachten wir folgendes Beispiel. 

\ex{Offene $\epsilon$-Kugel}
{
    Sei $(X,d)$ ein metrischer Raum und $O$ die Menge aller offenen Teilmengen von $X$,
    wobei $O \subseteqq X$ ist offen, wenn $\forall x \in O \exists \varepsilon > 0$ so dass die $\varepsilon$-Kugel
    $$
        U_\varepsilon(x) = \{ y \in X : d(y,x) < \varepsilon \}
    $$
    ganz in $O$ enthalten ist. 
    Die Menge $O \subseteq \mathcal{P}(X)$ hat die Eigenschaften:
    \begin{enumerate}
        \item $\emptyset, X \in O$,
        \item $O_1, O_2 \in O \implies O_1 \bigcap O_2 \in O$,
        \item $O_i \in O, i \in I$ Indexmenge beliebig $\implies \bigcup_{i \in I} O_i \in O$.
    \end{enumerate}
    Wörtlich bedeutet (1), dass sowohl die leere Menge als auch der ganze Raum
    offen sind. (2), dass der Schnitt zweier offener Mengen wieder offen ist.
    Und (3) dass die Vereinigung beliebig vieler offener Mengen wieder offen ist.
}

Diese Eigenschaften bilden den Ausgangspunkt unserer Verallgemeinerung. Und wir erhalten die folgende Definition.


\dfn{Topologie}
{
    Sei $X$ eine nichtleere Menge. Eine Menge $\mathcal{T} \subseteq \mathcal{P}(X)$ von Teilmengen 
von $X$ heißt \emph{Topologie} auf $X$, wenn $\mathcal{T}$ folgende Eigenschaften hat.
\begin{itemize}
    \item[(O1)] $\emptyset \in \mathcal{T}, \; X \in \mathcal{T}$.
    \item[(O2)] Aus $O_1, O_2 \in \mathcal{T}$ folgt $O_1 \bigcap O_2 \in \mathcal{T}$.
    \item[(O3)] Aus $O_i \in \mathcal{T}, \; i \in I$, mit einer beliebigen Indexmenge $I$ folgt 
    $\bigcup_{i \in I} O_i \in \mathcal{T}$.
\end{itemize}
Die Elemente von $\mathcal{T}$ heißen \emph{offene Mengen}. Das Paar $(X, \mathcal{T})$ 
bezeichnet man als \emph{topologischen Raum}.
}
\bigskip
\ex{Topologien}
{
\begin{itemize}
    \item[(i)] \textbf{Diskrete Topologie:} Sei $X \neq \emptyset$ und $\mathcal{T} = \mathcal{P}(X)$.
    \item[(ii)] \textbf{Indiscrete (Klumpen) Topologie:} $\mathcal{T} = \{\emptyset, X\}$.
    \item[(iii)] \textbf{cofinite Topologie:} Sei $X \neq \emptyset$ und  $\mathcal{T}, \mathcal{O} \subseteq \mathcal{P}(X)$ 
    definiert durch
$$
\mathcal{T} := \{ A \subseteq X : A = \emptyset \;\text{oder}\; X \setminus A \;\text{endlich} \}
\quad \text{und} \quad
\mathcal{O} := \{ A \subseteq X : A = X \;\text{oder}\; A \;\text{endlich} \}.
$$
\end{itemize}
    \nt{Wir überprüfen das die cofinite Topologie eine Topologie ist.\\
    Um das eizuhsehen muss man kurz hinschauen. Im Vergleich dazu muss 
    man das bei der diskreten und indiscrete Topologie nicht machen.
    \begin{itemize}
        \item[(O1)] Sei $V \subset \mathcal{T}$:
        \begin{enumerate}
            \item[1. Fall] $Y = \emptyset: \bigcup V = \emptyset$ 
            \item[2. Fall] $V = \emptyset: \bigcup V = \emptyset$
        Wähle $O \in V$ dann gilt $X\setminus \bigcap V \subset X \setminus O$ endlich. 
        \footnote{$\bigcup V := \{x \mid \forall v \in \mathcal{V}: x \in V\}$}
        \end{enumerate}
        \item[(O2)] Sei $V \in \mathcal{T}$ endlich, dann gilt $V = \emptyset: \bigcap V = X$
        \item[(O3)] Sei $V + \emptyset$, dann gilt $X \setminus \bigcup V = \underset{\text{endlich}}{\bigcap}$. ist endlich
    \end{itemize}
    }
}

\ex{Eine Metrik induziert eine Topologie}
{
Sei $(X,d)$ ein metrischer Raum. mit $X \times X \to \mathbb{R}$ Dann ist $U_{r}(x) = \{y \in X : d(x,y) < r\}$
eine offene Menge für alle $x \in X$ und $r > 0$. Wir definieren darauf die Topologie 
$$\mathcal{T} = \{ O \subseteq X \mid \forall x \in O \exists r > 0 : U_r(x) \subseteq O \}.$$  
Um einzusehen, dass $\mathcal{T}$ eine Topologie ist, müssen wir die drei Eigenschaften überprüfen bzw. schon zwei mal hinschauen.
\begin{itemize}
    \item[(O1)] $\emptyset, X \in \mathcal{T}$.
    \item[(O2)] Sei $V \in \mathcal{T}, x\in \bigcup V$. Dann wähle $V \in \mathcal{V}$ 
    und $r > 0$ so dass $U_r(x) \subseteq V$. Dann gilt auch $U_r(x) \subseteq V \subset \bigcup V$.
    \item[(O3)] Sei $\mathcal{V} \subset \mathcal{T}$ endlich, $x \in \bigcap V$. Dann für jedes $V \in \mathcal{V}$ wähle
    $r_V > 0$ so dass $U_{r_V}(x) \subseteq V$. 
    Dann gilt auch $U_{r_V}(x) \subseteq \bigcap V \Rightarrow \underset{V \in \mathcal{V}}{\bigcap} U_{r_V}(x) \subseteq \bigcap V$.
    Das noch etwas genauer aufgetröselt: $\underset{V \in \mathcal{V}}{\bigcap}\{y \in X \mid d(x,y) < r_V\} 
    = \{y \in X \mid \forall V \in \mathcal{V}: d(x,y) < r_V\}$.
\end{itemize}
}

\ex{}
{
Betrachte $\mathbb{Z}$. Für $k \in \mathbb{Z}$ und $l \ge 1$ sei $U_{k,l} := \{k+nl \mid n \in \mathbb{Z}\}$. 
Weiters definiere $\mathcal{T} := \{ k \in \mathbb{Z} \mid \forall x \in = \exists l \in \mathbb{Z}, l \ge ; U_{k,l} \subset 0\}$.
Dann ist $\mathcal{T}$ eine Topologie auf $\mathbb{Z}$.
\begin{itemize}
    \item[(O1)] $\emptyset, \mathbb{Z} \in \mathcal{T}$.
    \item[(O2)] Sei $V \subset \mathcal{T}$, $y\in \bigcup V$. 
    Dann wähle $V \in \mathcal{V}$ und $x \in \mathbb{Z}, l \ge 1$ so dass $U_{x,l} \subseteq V$.
    Dann gilt auch $U_{x,l} \subseteq V \subset \bigcup V$.
    \item[(O3)] Sei $V \subset\mathcal{B} endlich x \in \bigcap V$ Ein $V \in \mathcal{V}$ 
    wähle $l_v \in \mathbb{Z}. l_v \ge 1$ so dass $U_{x,l_v} \subseteq V$.
\end{itemize}
}