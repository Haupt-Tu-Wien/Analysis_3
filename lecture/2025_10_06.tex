\chapter{2025 10 06}
Da wir für viele Themen die in Analysis 3 behandelt werden, Resultate aus der Mastheorie benötigen,
fangen wir mit einigen Grundlagen der Topologie an. Da man für Thopologie nix braucht (man muss nur ein bisschen mit Mengen spielen).
\section{Topologische Grundbegriffe}

Im Folgenden geht es darum den Konvergenz Begriff zu verallgemeinern. 
Dazu betrachten wir Räume, gerade noch soviel Struktur tragen, dass man von stetigen
Funktionen, Grenzwerten, Kompaktheit und so weiter sprechen kann.

Zur wiederholung betrachten wir folgendes Beispiel. 

\ex{Offene $\epsilon$-Kugel}
{
    Sei $(X,d)$ ein metrischer Raum und $O$ die Menge aller offenen Teilmengen von $X$,
    wobei $O \subseteqq X$ ist offen, wenn $\forall x \in O \exists \varepsilon > 0$ so dass die $\varepsilon$-Kugel
    $$
        U_\varepsilon(x) = \{ y \in X : d(y,x) < \varepsilon \}
    $$
    ganz in $O$ enthalten ist. 
    Die Menge $O \subseteq \mathcal{P}(X)$ hat die Eigenschaften:
    \begin{enumerate}
        \item $\emptyset, X \in O$,
        \item $O_1, O_2 \in O \implies O_1 \cap O_2 \in O$,
        \item $O_i \in O, i \in I$ Indexmenge beliebig $\implies \bigcup_{i \in I} O_i \in O$.
    \end{enumerate}
    Wörtlich bedeutet (1), dass sowohl die leere Menge als auch der ganze Raum
    offen sind. (2), dass der Schnitt zweier offener Mengen wieder offen ist.
    Und (3) dass die Vereinigung beliebig vieler offener Mengen wieder offen ist.
}
Diese Eigenschaften bilden den Ausgangspunkt unserer Verallgemeinerung. Und wir erhalten die folgende Definition.


\dfn{Topologie}
{
    Sei $X$ eine nichtleere Menge. Eine Menge $\mathcal{T} \subseteq \mathcal{P}(X)$ von Teilmengen 
von $X$ heißt \emph{Topologie} auf $X$, wenn $\mathcal{T}$ folgende Eigenschaften hat.
\begin{itemize}
    \item[(O1)] $\emptyset \in \mathcal{T}, \; X \in \mathcal{T}$.
    \item[(O2)] Aus $O_1, O_2 \in \mathcal{T}$ folgt $O_1 \cap O_2 \in \mathcal{T}$.
    \item[(O3)] Aus $O_i \in \mathcal{T}, \; i \in I$, mit einer beliebigen Indexmenge $I$ folgt 
    $\bigcup_{i \in I} O_i \in \mathcal{T}$.
\end{itemize}
Die Elemente von $\mathcal{T}$ heißen \emph{offene Mengen}. Das Paar $(X, \mathcal{T})$ 
bezeichnet man als \emph{topologischen Raum}.
}

\ex{Topologien}
{
\begin{itemize}
    \item[(i)] \textbf{Diskrete Topologie:} Sei $X \neq \emptyset$ und $\mathcal{T} = \mathcal{P}(X)$.
    \item[(ii)] \textbf{Indiscrete (Klumpen) Topologie:} $\mathcal{T} = \{\emptyset, X\}$.
    \item[(iii)] \textbf{cofinite Topologie:} Sei $X \neq \emptyset$ und  $\mathcal{T}, \mathcal{O} \subseteq \mathcal{P}(X)$ 
    definiert durch
$$
\mathcal{T} := \{ A \subseteq X : A = \emptyset \;\text{oder}\; X \setminus A \;\text{endlich} \}
\quad \text{und} \quad
\mathcal{O} := \{ A \subseteq X : A = X \;\text{oder}\; A \;\text{endlich} \}.
$$
\end{itemize}

}