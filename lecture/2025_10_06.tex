\chapter{}
\section{Organisatorisches}
Alle Infomationen sind Online zu finden aber für die form von der Mitschrift sollte hier etwas stehen.:
Die LVa ist eine UE + Vo. 
\subsection{Übung}
Für das Positive absolvieren der Übung benötigt man ca. 60\% der Käuze und zwei positive Tafeleistungen.
Man darf einmal fehlen + nachbringen aber ohne guten Grund.
Für weiteres fehlen braucht man einen guten Grund (meine Oma ist gestoreben,
ist ab dem zweiten mal kein guter Grund mehr).
\subsection{Vorlesung}
Die Vorlesung ist ein Tafelvortag. 
Prüfung teilt sich auf in schriftlich und mündlich. Die schriftliche Prüfung ist als Ticket für die Mündliche zu verstehen.
Der stoff der Schriftlichen prüfug sind die Übungsbeispiel.
Der Stoff der Mündlichen Prüfung ist der Inhalt der Vorlesung (Wichtig hierbei der Vorlesung heißt nicht ihrgendeine Vorlesung).

\section{Topologische Grundbegriffe} 

Da wir für viele Themen die in Analysis 3 behandelt werden, Resultate aus der Mastheorie benötigen,
fangen wir mit einigen Grundlagen der Topologie an. Da man für Topologie nix braucht (man muss nur ein bisschen mit Mengen spielen).

\bigskip

Im Folgenden geht es darum den Konvergenz Begriff zu verallgemeinern. 
Dazu betrachten wir Räume, gerade noch soviel Struktur tragen, dass man von stetigen
Funktionen, Grenzwerten, Kompaktheit und so weiter sprechen kann.

Zur wiederholung betrachten wir folgendes Beispiel. 

\ex{Offene $\epsilon$-Kugel}
{
    Sei $(X,d)$ ein metrischer Raum und $O$ die Menge aller offenen Teilmengen von $X$,
    wobei $O \subseteqq X$ ist offen, wenn $\forall x \in O \exists \varepsilon > 0$ so dass die $\varepsilon$-Kugel
    $$
        U_\varepsilon(x) = \{ y \in X : d(y,x) < \varepsilon \}
    $$
    ganz in $O$ enthalten ist. 
    Die Menge $O \subseteq \mathcal{P}(X)$ hat die Eigenschaften:
    \begin{enumerate}
        \item $\emptyset, X \in O$,
        \item $O_1, O_2 \in O \implies O_1 \bigcap O_2 \in O$,
        \item $O_i \in O, i \in I$ Indexmenge beliebig $\implies \bigcup_{i \in I} O_i \in O$.
    \end{enumerate}
    Wörtlich bedeutet (1), dass sowohl die leere Menge als auch der ganze Raum
    offen sind. (2), dass der Schnitt zweier offener Mengen wieder offen ist.
    Und (3) dass die Vereinigung beliebig vieler offener Mengen wieder offen ist.
}

Diese Eigenschaften bilden den Ausgangspunkt unserer Verallgemeinerung. Und wir erhalten die folgende Definition.


\dfn{Topologie}
{
    Sei $X$ eine nichtleere Menge. Eine Menge $\mathcal{T} \subseteq \mathcal{P}(X)$ von Teilmengen 
von $X$ heißt \emph{Topologie} auf $X$, wenn $\mathcal{T}$ folgende Eigenschaften hat.
\begin{itemize}
    \item[(O1)] $\emptyset \in \mathcal{T}, \; X \in \mathcal{T}$.
    \item[(O2)] $\forall n \in \mathbb{N} und O_1, \dots O_n \in \mathcal{T} : 
     \bigcap_{i=1}^{n} O_i \in \mathcal{T}$.
    \item[(O3)] Aus $O_i \in \mathcal{T}, \; i \in I$, mit einer beliebigen Indexmenge $I$ folgt 
    $\bigcup_{i \in I} O_i \in \mathcal{T}$.
\end{itemize}
Die Elemente von $\mathcal{T}$ heißen \emph{offene Mengen}. Das Paar $(X, \mathcal{T})$ 
bezeichnet man als \emph{topologischen Raum}.
}
\nt{ Die Bedinnung (O2) kann auch abgespeckter formuliert werden. 
Es reicht zu fordern, dass
$$
\forall O_1, O_2 \in \mathcal{T} : O_1 \bigcap O_2 \in \mathcal{T}
$$ 
gilt. 
Dann folt die allgemeine Formulierung durch Vollständige Induktion.}

\ex{Topologien}
{
\begin{itemize}
    \item[(i)] \textbf{Diskrete Topologie:} Sei $X \neq \emptyset$ und $\mathcal{T} = \mathcal{P}(X)$.
    \item[(ii)] \textbf{Indiscrete (Klumpen) Topologie:} $\mathcal{T} = \{\emptyset, X\}$.
    \item[(iii)] \textbf{cofinite Topologie:} Sei $X \neq \emptyset$ und  $\mathcal{T}, \mathcal{O} \subseteq \mathcal{P}(X)$ 
    definiert durch
$$
\mathcal{T} := \{ A \subseteq X : A = \emptyset \;\text{oder}\; X \setminus A \;\text{endlich} \}
\quad \text{und} \quad
\mathcal{O} := \{ A \subseteq X : A = X \;\text{oder}\; A \;\text{endlich} \}.
$$
\end{itemize}
\begin{proof}
Die Beispiele sind so schon sehr schön aber noch schöner sind sie wenn wir nicht nur glauben 
müssen das es Topologien sind sondern es auch überprüfen können.
\begin{itemize}
    \item[(i)] Diskrete Topologie: Überprüfung durch hinsehen.
    \item[(ii)] Indiscrete Topologie: Überprüfung durch hinsehen.
    \item[(iii)] cofinite Topologie: überprüfen durch genauer hinschauen 
    \begin{enumerate}
        \item [(O1)] $\emptyset, X \in \mathcal{T}$.
        \begin{enumerate}
            \item[1. Fall] $Y = \emptyset: \bigcup V = \emptyset$ 
            \item[2. Fall] $V = \emptyset: \bigcup V = \emptyset$
        Wähle $O \in V$ dann gilt $X\setminus \bigcap V \subset X \setminus O$ endlich. 
        \footnote{$\bigcup V := \{x \mid \forall v \in \mathcal{V}: x \in V\}$}
        \end{enumerate}
        \item[(O2)] Sei $V \in \mathcal{T}$ endlich, dann gilt $V = \emptyset: \bigcap V = X$
        \item[(O3)] Sei $V + \emptyset$, dann gilt 
        $X \setminus \bigcup V = \underset{\text{endlich}}{\bigcap}$ ist endlich.
    \end{enumerate}
\end{itemize}

Damit haben wir (Das wir impliziert das es eine art Gruppen arbeit ist. 
Diese Implikation ist grundlegend falsch was jedem beusucher einer Vorlesung klar sein sollte.)
gezeigt das die cofinite Topologie Tatsächlich eine 
Topologie ist.

\end{proof}
}


Wir witmen uns nun noch weiteren Beispielen für Topologien. 

\ex{Eine durch eine Metrik induziert Topologie}
{
Sei $\langle X,d \rangle $ ein metrischer Raum. mit $X \times X \to \mathbb{R}$ Dann ist $U_{r}(x) = \{y \in X : d(x,y) < r\}$
eine offene Menge für alle $x \in X$ und $r > 0$. Wir definieren darauf die Topologie 
$$\mathcal{T}_d = \{ O \subseteq X \mid \forall x \in O \exists r > 0 : U_r(x) \subseteq O \}.$$  

Auch hier wollen wir noch zeigen das es sich tatsächlich um eine Topologie handelt.
\begin{proof}
Um einzusehen, dass $\mathcal{T}_d$ eine Topologie ist, müssen wir wieder 
die drei Eigenschaften überprüfen bzw. schon zwei mal hinschauen.\dots oder dreimal:
\begin{itemize}
    \item[(O1)] $\emptyset, X \in \mathcal{T}_d$.
    Wobei 
    $$
    X \subseteq X, \quad 
    \forall x \in X, \; r > 0 : \{ y \in X \mid d(x,y) < r \} \subseteq X
    $$
    Hierbei ist interessant zu bermerken das die leere Menge eine Teilmenge jeder Menge ist, 
    also insbesondere auch eine Teilmenge von der Umgebung $U_r(x)$.
    Da Eine Menge der Form $U_r(x)$ nie leer sein kann, da $x \in X$ und $d(x,x) = 0 < r$, 
    Das heist das $U_r(x)$ immer zumindest das Element $x$ selbst enthält. 

    \item[(O2)] Sei $V \in \mathcal{T}_d, x\in \bigcup V$. Dann wähle $V \in \mathcal{V}$ 
    und $r > 0$ so dass $U_r(x) \subseteq V$. Dann gilt auch 
    $$
    U_r(x) \subseteq V \subset \bigcup V
    $$.
    \item[(O3)] Sei $\mathcal{V} \subset \mathcal{T}_d$ endlich, $x \in \bigcap V$.
    Dann wähle für jedes $V \in \mathcal{V}$ ein $r_V > 0$ so dass
    $U_{r_V}(x) \subseteq V$\\
    Dann gilt auch 
    $$
    U_{r_V}(x) \subseteq \bigcap V \Rightarrow \underset{V \in \mathcal{V}}{\bigcap} U_{r_V}(x) \subseteq \bigcap V
    $$
    Das noch etwas genauer aufgetröselt: 
    $$
    \underset{V \in \mathcal{V}}{\bigcap}\{y \in X \mid d(x,y) < r_V\} 
    = \{y \in X \mid \forall V \in \mathcal{V}: d(x,y) < r_V\}
    $$.
\end{itemize}
\end{proof}
}

\ex{Geometrische Regression}
{
Betrachte $\mathbb{Z}$. Für $k \in \mathbb{Z}$ und $l \ge 1$ sei $U_{k,l} := \{k+nl \mid n \in \mathbb{Z}\}$. 
Weiters definiere $\mathcal{T} := \{ k \in \mathbb{Z} \mid \forall x \in = \exists l \in \mathbb{Z}, l \ge ; U_{k,l} \subset 0\}$.
Dann ist $\mathcal{T}$ eine Topologie auf $\mathbb{Z}$.
\begin{itemize}
    \item[(O1)] $\emptyset, \mathbb{Z} \in \mathcal{T}$.
    \item[(O2)] Sei $V \subset \mathcal{T}$, $y\in \bigcup V$. 
    Dann wähle $V \in \mathcal{V}$ und $x \in \mathbb{Z}, l \ge 1$ so dass $U_{x,l} \subseteq V$.
    Dann gilt auch $U_{x,l} \subseteq V \subset \bigcup V$.
    \item[(O3)] Sei $V \subset\mathcal{B} endlich x \in \bigcap V$ Ein $V \in \mathcal{V}$ 
    wähle $l_v \in \mathbb{Z}. l_v \ge 1$ so dass $U_{x,l_v} \subseteq V$.
\end{itemize}
}