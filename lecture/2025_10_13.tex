\chapter{}
Ein großes Problem in der Mathematik ist die Benennung von Objekten.
Hier wird einem diese Problem deutlich vor Augen geführt.
Um eine kleine Orientierung zu bieten lege ich jetzt fest, und hoffe, dass
ich mich auch daran halte, welche Schriftarten ich für welche Objekte festlegen
möchte.
\nt{
Notation:
\begin{itemize}
    \item Standart Menge:  $A,B,C, \dots X$
    \item Umgebungen (Spezifischere Mengen): $\mathscr{A}, \mathscr{B}, \mathscr{C}, \dots \mathscr{X}$
    \item Topologie (Menge mit Offenen Mengen): $\mathcal{A}, \mathcal{B}, \mathcal{C}, \dots \mathcal{X}$
    \item Filter (Menge mit bestimmten Eigenschaften): $\mathfrak{A}, \mathfrak{B}, \mathfrak{C}, \dots \mathfrak{X}$
    \item Punkte (Elemente der Menge); $a,b,c, \dots x$
\end{itemize}
}

Wir haben ( leider\footnote{Die Mathematik wäre möglicherweise einfacher mit weniger} ) 
noch mehr Objekte, und mehr Schriftarten würden nicht zur besseren
Übersichtlichkeit beitragen.  - Ich persönlich weiß auch das ich früher 
oder noch früher durcheinander kommen werde wenn ich neben:
mathcal{}, mathbf{} mathfrak{}, mathbb{}, mathscr{} noch mehr schriftarten verwende.-
\footnote{ Es sind so schon zu viele für meinen Geschmack.}
\footnote{$\mathcal{S} \mathbf{S} \mathfrak{S} \mathbb{S} \mathscr{S}$} 


\dfn{Dichtheit, Häufungspunkt, Isolierter Punkt}
{
    Sei $\langle X, \mathcal{T}\rangle$ ein Topologischer Raum und $A \subset X$.
\begin{itemize}
    \item $A$ ist \textbf{dicht} in X, falls $\bar{A} = X$.
    \item $x \in X$ ist ein \textbf{Häufungspunkt} von $A$, falls:
     $\forall U \in \mathfrak U(x) \exists y \in A : y \in U \setminus \{ x \}$.
    \item $x \in X$ ist ein \textbf{isolierter Punkt} von $A$, falls:
     $\exists 0 \in \mathcal{T} : x \in O : 0 \cap A = \{x\}$.
\end{itemize}
}

\dfn{Umgebung}
{
    Sei $\langle X, \mathcal{T}\rangle$ ein Topologischer Raum und $x \in X$.
    Eine Menge $\mathcal{U} \subset X$ heißt \textbf{Umgebung} von $x$, falls
    $$
    \exists O \in \mathcal{T} : x \in O \subset \mathcal{U}
    $$.
}

\dfn{Filter}
{
   Für eine nichtleere Menge $M$ heißt $\mathfrak{F} \subseteq \mathcal{P}(M)$ 
   ein \emph{Filter} auf $M$, wenn
\begin{enumerate}
  \item[(F1)] $\mathfrak{F} \neq \emptyset$ und $\emptyset \notin \mathfrak{F},$
  \item[(F2)] $F_1 \cap F_2 \in \mathfrak{F}$, falls $F_1, F_2 \in \mathfrak{F},$
  \item[(F3)] $F_2 \in \mathfrak{F}$, falls $F_1 \in \mathfrak{F}$
   und $F_1 \subseteq F_2 \subseteq M.$
\end{enumerate} 
}

\mlenma{}{

  Sei $\langle X, \mathcal{T}\rangle$ ein topologischer Raum und 
  $x \in X$. Dann ist die Menge aller Umgebungen 
  $$
  \mathscr{U}(x) = \{ U \in \mathcal{T} \mid x \in U \}
  $$
  von $x$ ein Filter auf $X$.
}
\begin{proof}{Lemma: 3.0.4}\\
Wir überprüfen die Filtereigenschaften.:
    
    \begin{itemize}
    \item[(F1)]$x \in \mathscr{U}(x)$ und $X \in \mathcal{T}, x \in X \subseteq X$ 
    Damit haben wir
          $$
          \emptyset \notin \mathscr{U}(x) \text{ und } 
          \forall U \in \mathscr{U}(x): x \in U
          $$
    \item[(F2)] $U_1, U_2 \in \mathcal{U}(x) \ \forall O_1, O_2 \in \mathcal{O} 
    \text{ mit } x \in O_1 \subseteq U_1, \ x \in O_2 \subseteq U_2$ \\
          $$
          \Rightarrow x \in O_1 \cap O_2 \subseteq U_1 \cap U_2 \\
          \Rightarrow U_1 \cap U_2 \in \mathcal{U}(x)
          $$
    \item[(F3)] $U \in \mathscr{U}(x), V \supseteq$ Wähle 
        $$
        O \in \mathcal{T} \text{ und } x \in O \subseteq \mathscr{U} \subseteq V
        $$
    \end{itemize}
    
\end{proof}

Hir ist meine Notationelle Einführung schon an seine Grenze gekommen. 
Wie nehmen die Menge (also U) aller Umgebungen von $x$ und nennen sie $\mathscr{U}(x)$.
aber wir zeigen, dass es ein Filter ist. Damit wäre ja $\mathfrak{U}(x)$ 
die Richtige Notation.
- Ich will konsistent bleiben aber habe es schon aufgegeben. -

\mprop{}
{
    Sei $\langle X, \mathcal{T} \rangle$ Top-Raum 
    und $\langle Y, \mathcal{V} \rangle$ Top-Raum
    Dann gilt:
\begin{enumerate}
    \item $\forall O \in X \Leftrightarrow (\forall x \in O: O \in \mathscr{U}(x))$
    \item Sei $f : X \to Y$, dann gilt: $f$ stetig $\Leftrightarrow$
    $$
    \forall x \in X  \forall V \in \mathcal{V}(f(x)) \in \mathscr{U}^X(f(x))
    \exists u \in \mathscr{U}^Y(f(x)): f(u) \subseteq \mathscr{U}^Y(f(U)) 
    \subseteq V
    $$
    \item Sei $A \in X$, dann ist $\overline{A} = \{ x \in X \mid \forall u
     \in \mathscr{U}(x) : U \cap A \neq \emptyset \}$
\end{enumerate}
}

\begin{proof}{Prop: 3.0.5}\\
\begin{enumerate}
    \item 
    \begin{itemize}
        \item["$\Rightarrow$"] 
        Sei $x \in O \in \mathcal{T}$, dann gilt:
        $$
        x \in O \subseteq 0
        $$
        \item["$\Leftarrow$"] 
        $\forall x \in O$ wähle $\mathscr{Q}_x \subseteq O$ 
        Dann haben wir:
        $$
            \{x\} \subseteq O_x \subseteq O\\
             \Rightarrow 
            \bigcup_{x \in O} \{x\} \subseteq 
            \underbrace{\bigcup_{x \in O} \mathscr{Q}_x}_{\text{ Offen}} \subseteq 
            \bigcup_{x \in O} O = O
        $$
    \end{itemize}
    \item    
    \begin{itemize}
        \item["$\Rightarrow$"] 
        Sei $x \in X$, $V \in \mathscr{U}^Y(f(x))$. Wähle $Q \in \mathfrak{V}$ \\
        mit $f(x) \in Q \subseteq V$ \\
        $\Rightarrow f^{-1}(Q) \in \mathcal{T}$  Das Folgt aus unseer Charakterisierung
        von stetigen Funktionen. \\
        Betrachte $x \in f^{-1}(Q) \in \mathscr{U}^X(x)$ damit haben wir:
        $$
        f(f^{-1}(Q)) \subseteq Q \subseteq \mathfrak{V}
        $$
        \item["$\Leftarrow$"]
        Sei $O \in \mathfrak{V}$ 
        Wir wollen zeigen dass, das Vollständige Urbild wieder offen ist.
        Sei $x \in f^{-1}(O)$, d.h. ($f(x) \in O \Rightarrow O 
        \in \mathscr{U}^Y(f(x))$) \\
        Wähle $U \in \mathscr{U}^X(x)$ mit $f(U) \subseteq O$ \\
        $$\Rightarrow U \subseteq f^{-1}(f(U)) \subseteq f^{-1}(O)$$
        und 
        $$U\in \mathscr{U}^X(x) \Rightarrow f^{-1}(O) \in \mathscr{U}^X(x)
        \Rightarrow f^{-1}(O) \in \mathcal{T}$$
    \end{itemize}
    \item
    \begin{itemize}
        \item["$\subseteq$"]
        Sei $x \notin \overline{A}$, d.h. 
        $x \in X\setminus \overline{A}$ (Wobei $ X \setminus \overline{A}$) 
        ist offen $: \leftrightarrow \text{ Offen } \Leftrightarrow
        X\setminus \overline{A} \in \mathscr{U}(x)$ \\
        $$
        \overline{A} \supseteq A \Rightarrow X\setminus \overline{A} 
        \Rightarrow A \cap (X\setminus \overline{A}) = \emptyset
        $$
        \item["$\supseteq$"]
        Sei $x \in X$ und $\exists U \in \mathscr{U}(x) :U \cap A = \emptyset$ 
        wähle $O \in \mathcal{T}$ mit $x \in O \subseteq U$ \\
        $$
        \Rightarrow O \cap A = \emptyset \Rightarrow A \subseteq X \setminus O
        $$ 
        $$
        \Rightarrow \overline{A} \subseteq X \setminus O \Rightarrow x \in \overline{A}
        $$
    \end{itemize}
\end{enumerate}
\end{proof}

\mlenma{}
{
    \begin{itemize}
        \item $\langle X, \mathcal{T} \rangle$ Topologischer-Raum, dann ist $id_X$ stetig.
        \item $\langle X, \mathcal{T} \rangle$, $\langle Y, \mathcal{V} \rangle$
        und $\langle Z, \mathcal{W} \rangle$
        Topologische-Räume und $f : X \to Y$ und $g : Y \to Z$ stetig, dann ist
        $g \circ f : X \to Z$ stetig.
    \end{itemize}
}

\begin{proof}{Lemma: 3.0.6}\\
    \begin{itemize}
        \item $X \overset{id:X}{\to} X$ Sei $O \in \mathcal{T}$, dann gilt:
        $$
        id_X^{-1}(O) = O \in \mathcal{T}
        $$
        \item $X \overset{f}{\to} Y \overset{g}{\to} Z$ 
        Sei $O \in \mathcal{W}$, dann gilt:
        $$
        g^{-1}(O) \in \mathcal{V} \text{ und } f^{-1}(g^{-1}(O)) \in \mathcal{T}
        $$
        $$\Rightarrow (g \circ f)^{-1}(O) = f^{-1}(g^{-1}(O)) \in \mathcal{T}
        $$
        $$
        \Rightarrow g \circ f \text{ ist stetig.}
        $$
    \end{itemize}
\end{proof}

\nt{
Zur Wiederholung: Aus Analysis 1 wissen wir, dass jede 
Folge $(x_n)_{n \in \mathbb{N}}$ höchstens einen Grenzwert hat.

\begin{proof}{Eindeutigkeit des Grenzwerts}\\
Seien  $x_n \to x$ und $x_n \to y$ mit $x \neq y (\Leftrightarrow d(x,y) > 0)$.\\
Wähle $N_1, N_2\in \mathbb{N}$, sodass 
$d(x_n, x) < \frac{d(x,y)}{3} \forall n \geq N_1$, und
$d(x_n, y) < \frac{d(x,y)}{3}$ 
$ \forall n \geq N_2$.\\
Dann folgt für $N := \max\{N_1, N_2\}$ der Widerspruch:
$$
    d(x,y) \leq d(x, x_N) + d(x_N, y)
    < \frac{d(x,y)}{3} + \frac{d(x,y)}{3}
    = \frac{2\,d(x,y)}{3}
    < d(x,y).
$$
\end{proof}
}

 Im Folgenden betrachten wir Hausdorff-Räume. Das sind 
 Räume in denen sich Punkte trennen lassen 
 (Die T2-Eigenschaft bzw. Hausdorff-Eigenschaft). Diese Eigenschaft ist 
 im Grunde vergleichbar mit der Eindeutigkeit von Grenzwerten in
 metrischen Räumen.

\dfn{Hausdorff-Raum}
{
Ein topologischer Raum $\langle X, \mathcal{T} \rangle$
heißt \emph{Hausdorff-Raum} oder \emph{$(T2)$-Raum}, 
wenn folgendes Trennungsaxiom erfüllt ist:
\begin{center}
    \textbf{(T2)} \quad 
    Zu je zwei Punkten $x, y \in X$, $x \neq y$, gibt es disjunkte, offene Mengen 
    $O_x$ und $O_y$ mit $x \in O_x$ und $y \in O_y$.
\end{center}
}

Bevor wir weiter mit der Hausdorff-Eigenschaft arbeiten, betrachten wir
bzw definieren wir noch gerichtete Mengen.\\
\textbf{Die nebenbei viel mehr Sinn machen um das Riemann-Integral zu definieren.}

\dfn{Gerichtete Menge}
{
    Sei $I$ eine nicht leere Menge und $\leq$ eine Relation auf $I$. 
Dann heißt $(I, \leq)$ eine \textit{gerichtete Menge}, wenn $\leq$ folgenden drei Bedingungen genügt:

\begin{itemize}
    \item[\textbullet] \textbf{Reflexivität:} \\
    $$\forall i \in I : \; i \leq i.$$

    \item[\textbullet] \textbf{Transitivität:} \\
    $$\forall i, j, k \in I : \; (i \leq j \land j \leq k) \Rightarrow i \leq k.$$

    \item[\textbullet] \textbf{Richtungseigenschaft:} \\
    $$\forall i, j \in I \; \exists k \in I : \; i \leq k \land j \leq k.$$
\end{itemize}
}

Nach dem Wir jetzt wissen was eine gerichtete Menge ist, 
können wir uns auch gleich die Riemannsumme anschauen.

\ex{Riemansumme}
{
    Wir nennen das Paar 
$$
\mathcal{R} = \bigl((\xi_j)_{j=0}^{n(\mathcal{R})}, \, (\eta_j)_{j=1}^{n(\mathcal{R})}\bigr)
$$
eine \emph{Riemann-Zerlegung} eines Intervalls $[a,b]$, falls
$$
a = \xi_0 < \xi_1 < \dots < \xi_{n(\mathcal{R})} = b, 
\qquad \eta_j \in [\xi_{j-1}, \xi_j], \quad j=1,\dots,n(\mathcal{R}),
$$
und nennen
$$
|\mathcal{R}| := \max \{\, \xi_j - \xi_{j-1} \; ; \; j=1,\dots,n(\mathcal{R}) \,\}
$$
die \emph{Feinheit der Zerlegung}.

Weiter sei 
$$
\mathcal{R}_1 \preceq \mathcal{R}_2 \;:\Longleftrightarrow\; |\mathcal{R}_2| \leq |\mathcal{R}_1|
$$
Ist $\Re$ die Menge aller solcher Zerlegungen, dann ist $(\Re,\preceq)$ eine gerichtete Menge.
In diesem Beispiel ist $\preceq$ sicher nicht antisymmetrisch.

\medskip

Dieser gerichteten Menge und Jener aus dem letzten Beispiel werden wir bei der Einführung 
des Integrals wieder begegnen.
}

\mprop{}
{
    Sei $\langle X, \mathcal{T} \rangle$ ein Topologischer Raum.
    Dann sind folgende Aussagen äquivalent:
\begin{itemize}
    \item[(i)] $X$ ist ein Hausdorff-Raum.
    \item[(ii)] Jede Folge in $X$ hat höchstens einen Grenzwert
\end{itemize}
}

\begin{proof}{Prop: 3.0.9}\\
    \begin{itemize}
\item[(i) $\Rightarrow$ (ii)] 
    Wären $x, y$ zwei verschiedene Grenzwerte, so könnten wir disjunkte Umgebungen 
    $U \in \mathcal{U}(x)$ und $V \in \mathcal{U}(y)$ wählen und infolge 
    $i_1 \in I$ und $i_2 \in I$ derart wählen, dass 
    $$
    x_i \in U \quad \text{für alle } i \geq i_1
    \qquad \text{und} \qquad
    x_i \in V \quad \text{für alle } i \geq i_2.
    $$
    Da $I$ gerichtet ist, gibt es ein $i \in I$ mit $i \geq i_1$ und $i \geq i_2$, 
    und daher $x_i \in U \cap V$, was aber im Widerspruch zu 
    $U \cap V = \emptyset$ steht.

\item[(ii) $\Rightarrow$ (i)] Beweis durch Kontraposition.([$\neg$(ii) $\Rightarrow$ $\neg$(i)])\\
    \begin{itemize}
    \item    Wir definieren eine gerichtete Menge:\\
    Wähle $x, y \in X$ und $x \neq y$. \\
    Da $(X,\mathcal{T})$ Hausdorff ist, existieren offene Mengen 
    $O_x, O_y \in \mathcal{T}$ mit
    $$
    x \in O_x, \quad y \in O_y \quad \text{und} \quad O_x \cap O_y = \varnothing.
    $$

    Sei nun
    $$
    I := \left\{\, (O_1, O_2) \in \mathcal{T} \times \mathcal{T} \;\middle|\; 
    x \in O_1 \ \land \ y \in O_2 \,\right\}
    $$
    und definiere die Ordnung
    $$
    (O_1, O_2) \preceq (O_1', O_2') 
    \; :\Longleftrightarrow \;
    O_1' \subseteq O_1 \ \land \ O_2' \subseteq O_2.
    $$
    \item Damit ist $(I, \preceq)$ gerichtet.\\
    
    \begin{itemize}
        \item Reflexivität: 
        $$
        (O_x, O_y) \in I\text{ und }(O_x, O_y) \preceq (O_x, O_y)
        $$.
        \item Transitivität: Für $(O_1, O_2) \preceq (O_1', O_2')$ und 
        $(O_1', O_2') \preceq (O_1'', O_2'')$
        gilt offensichtlich auch 
        $$
        (O_1, O_2) \preceq (O_1'', O_2'')
        $$.
        \item Richtungseigenschaft: 
         Für $(O_1, O_2), (O_1', O_2') \in I$ ist
        $$
        (O_1 \cap O_2, \; O_1' \cap O_2') \in I
        $$
        ist gröber
    \end{itemize}
    \item Die Menge $\{O_1 \cap O_2 \mid (O_1, O_2) \in I \}$
    ist nicht leer, da $(x, y) \in I$.

    Wähle nun eine $\alpha : I \to \underset{(O_1, O_2) \in I}{\bigcup}$  eine Abbildung \\
    mit 
    $$
    \forall (O_1, O_2) \in I :\alpha(O_1, O_2) \in O_1 \cap O_2
    $$
    \footnote{Das ist die Stelle an der das Auswahlaxiom verwendet wird. Also lobpreiset das Auswahlaxiom.
    (Das Auswahlaxiom ist in der Zermelo-Fraenkel-Mengenlehre nicht beweisbar, aber auch nicht widerlegbar.)}

    \item Es fehlt noch zu zeigen, dass $\alpha$ ein konvergierendes Netz ist.\\
    $$
    \lim_{(O_1,O_2)} \alpha(O_1,O_2) = x .
    $$

    Sei $U \in \mathscr{U}(x)$, wähle $O \in \mathcal{T}$ : $x \in O \subseteq U$

    $$
    \forall (O_1, O_2) \in I : \exists \ O \Rightarrow (O_1, O_2) \succeq (O, O_x) 
    \Rightarrow O_1 \cap O_2 \subseteq O_1 \subseteq O \footnote{Das geht wegen dem Auswahl axiom}
    $$
    mit $i_0 := (O, x)$\\
    Auf der anderen Seite haben wir: \\
    $\forall (O_1, O_2) \in I : O_2 \subseteq O 
    \quad \text{(Das selbe spiel umgekehrt)} $

    Sei $U \in \mathscr{U}(y)$, wähle $O \in \mathcal{T} : y \in O \subseteq U$

    $$
    \forall (O_1, O_2) \in I : O_2 \subseteq O 
    \Rightarrow \alpha((O_1, O_2)) \in O_1 \cap O_2 \subseteq O_2 \subseteq O \subseteq U
    $$

\end{itemize}
\end{itemize}
\end{proof}

Die Richtung (ii) $\Rightarrow$ (i) hat keinen wirklichen 
Mehrwert - Ausser dass wir über zwei offene Mengen indexiert haben (WOW)-.
Wir haben diese Richtung also vorallem dem Beweis zu liebe gemacht.
