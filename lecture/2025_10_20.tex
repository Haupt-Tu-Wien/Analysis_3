\chapter{}


\thm{Charakterisierung der Stetigkeit}
{
    Seien $\langle X, \mathcal{t} \rangle$ und 
    $\langle Y, \mathcal{V}\rangle$
    Topologische Räume und $f : X \to Y$
    \begin{equation*} 
    \begin{split}
        f \text{ ist stetig } :
        &\Leftrightarrow \forall O \in \mathcal{V} 
        : f^{-1}(O) \in \mathcal{T}\\
        &\leftrightarrow  \forall A \subseteq Y \text{ abgeschlossen } 
        : f^{-1}(A) \text{ ist Abgeschlossen }\\
        &\leftrightarrow \forall x \in X \forall \mathcal{U} 
        \in \mathscr{U}^{Y}(f(x)) :
        \exists v \in \mathcal{U}^{X}(x) : f(V) \in \mathscr{U}\\
        &\leftrightarrow \text{ Für jedes Netz } \phi \in X,
        x \in X: x \text{ ist Grenzwert von } 
        \phi \Rightarrow f(x) \text{ ist Grenzwert von } f \circ \phi
    \end{split}
    \end{equation*}
}

In der letzten Vorlesung haben wir die Hinrichtung von der driten equivalenz schon im zuge vom beweis von \textbf{4.0.5} schon gesehen es fehlt also noch 
die Hinrichtung in die andere Richtung \\
- Gängiger wäre Rückrichtung - \\
zu zeigen\footnote{Aber Hinrichtung
hört sich lustiger an und passt auch besser zum Mathematik studium}. 

\begin{proof}{Satz 5.0.1:}\\
    Seien $\langle I, \preccurlyeq \rangle$ eine gerichtete Menge und
    $\varphi : I \to X$ ein Netz in $X$ mit Grenzwert $x \in X$.\\
    Sei $U \in\mathcal{U}^{Y}(f(x))$
    Da  $f$  Stetig können wir per Definition ein $V$ wählen so dass

    \begin{equation*} 
    \begin{split}
    &V \in \mathcal{U}^{X}(x):f(V) \subseteq \mathcal{U}\\
    \text{ Weiters wähle } \quad &i_0 \in I: \forall i \succcurlyeq i_0: \varphi(i) \in V\\
    &\Rightarrow \forall i \succcurlyeq i_0: (f\circ \varphi)(i) \in f(V) \subseteq U
    \end{split}
    \end{equation*}
    
\end{proof}
\section{Initiale Topologie*}

Seien $(Y_i, \mathcal{V}_i)$ topologische Räume, X eine Menge
und $f_i : X \to Y_i$ Abbildungen.
Wir nennen die  größte Topologie, die alle $f_i$ stetig macht, 
die \textbf{initiale Topologie} bezüglich der $f_i$.
$$
\setlength{\unitlength}{1cm}
\begin{picture}(6,3.5)
  \put(0,1.7){$Y$}

  \put(0.4,1.8){\vector(3,1){3}}
  \put(0.5,1.7){\vector(1,0){3}}
  \put(0.4,1.6){\vector(3,-1){3}}

  \put(4,2.8){$(X_i, \mathcal{V}_i)$}
  \put(4,1.7){$(X_j, \mathcal{V}_j)$}
  \put(4,0.6){$(X_k, \mathcal{V}_k)$}

  \put(1.8,2.4){$f_i$}
  \put(1.8,1.85){$f_j$}
  \put(1.8,1.0){$f_k$}

  \put(4.3,2.35){$\vdots$}
  \put(4.3,1.15){$\vdots$}
\end{picture}
$$

Das bedeutet (vgl.: Kaltenbäck: Aufabau Analysis ) 
\begin{itemize}
    \item Unter allen Topologien $\mathcal{V}'$ auf $X$ so, dass 
    $f_i : (X, \mathcal{V}') \to (X_i, \mathcal{V}_i)$ für alle $i \in I$
    stetig ist, ist 
    $$
    \mathcal{V} := \mathcal{V}\!\left( \bigcup_{i \in I} f_i^{-1}(\mathcal{V}_i) 
    \right),
    $$
    also die Topologie mit $\bigcup_{i \in I} f_i^{-1}(\mathcal{V}_i)$ 
    als Subbasis, die größte.  
    Man bezeichnet $\mathcal{V}$ als \textbf{initiale Topologie} 
    bezüglich der Abbildungen $f_i$, $i \in I$.

    \item Für diese initiale Topologie $\mathcal{V}$ 
    und einen weiteren topologischen Raum $(Y, \mathcal{O})$ 
    ist eine Abbildung $f : (Y, \mathcal{O}) \to (X, \mathcal{V})$ 
    genau dann stetig, 
    wenn $f_i \circ f : (Y, \mathcal{O}) \to (X_i, \mathcal{V}_i)$ 
    für alle $i \in I$ stetig ist.
\end{itemize}

\section{Produkttopologie}




