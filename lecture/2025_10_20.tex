\chapter{}


\thm{Charakterisierung der Stetigkeit}
{
    Seien $\langle X, \mathcal{t} \rangle$ und 
    $\langle Y, \mathcal{V}\rangle$
    Topologische Räume und $f : X \to Y$
    \begin{equation*} 
    \begin{split}
        f \text{ ist stetig } :
        &\Leftrightarrow \forall O \in \mathcal{V} 
        : f^{-1}(O) \in \mathcal{T}\\
        &\leftrightarrow  \forall A \subseteq Y \text{ abgeschlossen } 
        : f^{-1}(A) \text{ ist Abgeschlossen }\\
        &\leftrightarrow \forall x \in X \forall \mathcal{U} 
        \in \mathscr{U}^{Y}(f(x)) :
        \exists v \in \mathcal{U}^{X}(x) : f(V) \in \mathscr{U}\\
        &\leftrightarrow \text{ Für jedes Netz } \phi \in X,
        x \in X: x \text{ ist Grenzwert von } 
        \phi \Rightarrow f(x) \text{ ist Grenzwert von } f \circ \phi
    \end{split}
    \end{equation*}
}

In der letzten Vorlesung haben wir die Hinrichtung von der driten equivalenz schon im zuge vom beweis von \textbf{4.0.5} schon gesehen es fehlt also noch 
die Hinrichtung in die andere Richtung \\
- Gängiger wäre Rückrichtung - \\
zu zeigen\footnote{Aber Hinrichtung
hört sich lustiger an und passt auch besser zum Mathematik studium}. 

\begin{proof}{Satz 5.0.1:}\\
    Seien $\langle I, \preccurlyeq \rangle$ eine gerichtete Menge und
    $\varphi : I \to X$ ein Netz in $X$ mit Grenzwert $x \in X$.\\
    Sei $U \in\mathcal{U}^{Y}(f(x))$
    Da  $f$  Stetig können wir per Definition ein $V$ wählen so dass

    \begin{equation*} 
    \begin{split}
    &V \in \mathcal{U}^{X}(x):f(V) \subseteq \mathcal{U}\\
    \text{ Weiters wähle } \quad &i_0 \in I: \forall i \succcurlyeq i_0: \varphi(i) \in V\\
    &\Rightarrow \forall i \succcurlyeq i_0: (f\circ \varphi)(i) \in f(V) \subseteq U
    \end{split}
    \end{equation*}
    
\end{proof}
\section{Initiale Topologie*}

Seien $(Y_i, \mathcal{V}_i)$ topologische Räume, X eine Menge
und $f_i : X \to Y_i$ Abbildungen.
Wir nennen die  größte Topologie, die alle $f_i$ stetig macht, 
die \textbf{initiale Topologie} bezüglich der $f_i$.
$$
\setlength{\unitlength}{1cm}
\begin{picture}(6,3.5)
  \put(0,1.7){$Y$}

  \put(0.4,1.8){\vector(3,1){3}}
  \put(0.5,1.7){\vector(1,0){3}}
  \put(0.4,1.6){\vector(3,-1){3}}

  \put(4,2.8){$(X_i, \mathcal{V}_i)$}
  \put(4,1.7){$(X_j, \mathcal{V}_j)$}
  \put(4,0.6){$(X_k, \mathcal{V}_k)$}

  \put(1.8,2.4){$f_i$}
  \put(1.8,1.85){$f_j$}
  \put(1.8,1.0){$f_k$}

  \put(4.3,2.35){$\vdots$}
  \put(4.3,1.15){$\vdots$}
\end{picture}
$$

Das bedeutet (vgl.: Kaltenbäck: Aufabau Analysis ) 
\begin{itemize}
    \item Unter allen Topologien $\mathcal{V}'$ auf $X$ so, dass 
    $f_i : (X, \mathcal{V}') \to (X_i, \mathcal{V}_i)$ für alle $i \in I$
    stetig ist, ist 
    $$
    \mathcal{V} := \mathcal{V}\!\left( \bigcup_{i \in I} f_i^{-1}(\mathcal{V}_i) 
    \right),
    $$
    also die Topologie mit $\bigcup_{i \in I} f_i^{-1}(\mathcal{V}_i)$ 
    als Subbasis, die größte.  
    Man bezeichnet $\mathcal{V}$ als \textbf{initiale Topologie} 
    bezüglich der Abbildungen $f_i$, $i \in I$.

    \item Für diese initiale Topologie $\mathcal{V}$ 
    und einen weiteren topologischen Raum $(Y, \mathcal{O})$ 
    ist eine Abbildung $f : (Y, \mathcal{O}) \to (X, \mathcal{V})$ 
    genau dann stetig, 
    wenn $f_i \circ f : (Y, \mathcal{O}) \to (X_i, \mathcal{V}_i)$ 
    für alle $i \in I$ stetig ist.
\end{itemize}

\section{Produkttopologie}


Betrachte eine Menge $X$ die nicht leer ist und eine Menge 
$Y_x$ mit $x \in X$. Damit definieren wir:
$$
\prod_{x \in X} Y_x := \left\{ f : X \to \bigcup_{x \in X} Y_x 
\mid \forall x \in X : f(x) \in Y_x \right\}.
$$

Weiters führen wir die Notation:
$$
Y_1 \times Y_2 := \{(y_1,y_2): y_1 \in Y_1, y_2 \in Y_2\}
$$
\nt
{
    Analog zu der Eingeführten Notation schreiben wir:\\
    Für ein Element $f \in \prod_{x \in X} Y_x$ schreiben wir auch als 
    Tupel $(y_x)_{x \in X}$ was $(f(x))_{x \in X}$ entspricht.
}

\dfn{Produkttopologie}
{
Sei $X$ eine nicht leere Menge, $(Y_x, x \in X)$ eine Familie von Mänen
die nicht leer sind, und für jedes $x \in X$ sei 
$(Y_x, \mathcal{T}_x)$ ein topologischer Raum. Wir betrachten:
$$
S:= \{\pi_x^{-1}(O_x) \mid x \in X, O_x \in \mathcal{T}_x\}
$$
Die von S erzeugte Topologie ist die \textbf{Produkttopologie} und
wir schreiben auch :$\prod_{x \in X} T_x$
}

\nt{
\dfn{Produkttopologie (Kaltenbäck)}
{
    Sei $X$ eine nicht leere Menge und für jedes $x \in X$ 
    sei $(Y_x, \mathcal{V}_x)$ ein topologischer Raum. 
    Die \textbf{Produkttopologie} auf 
    $\prod_{x \in X} Y_x$ ist die initiale Topologie bezüglich der 
    Projektionen 
    $$
    \pi_x : \prod_{x \in X} Y_x \to Y_x, 
    \quad (y_x)_{x \in X} \mapsto y_x,
    $$
    für alle $x \in X$.
}
}

\mprop{}
{
  Sei (I, $\preccurlyeq$) eine gerichtete Menge und 
  $\varphi : I \to \prod_{x \in X} Y_x$ ein Netz.
  Weiter sein $g \in \prod_{x \in X} Y_x$. Dann gilt:
  $$
  \lim_{i \in I} \varphi(i) = g \Leftrightarrow \forall x_0 \in X :
  \lim_{i \in I} (\pi_{x_0}\circ \varphi)(i) = \pi_{x_0}(g) 
  $$
  Was Punkteweise Konvergenz bedeutet.
}

\begin{proof}{Proposition: 5.2.3:}\\
    \begin{itemize}
        \item["$\Rightarrow$":]
        Die Richtung ist klar, da die Projektionen $(\pi_{x_{0}})$ stetig sind. 
        und stetigkeit erhält Konvergenz.Dh. es folgt aus der Definition
        von $S$.
        \item["$\Leftarrow$":]
        Sei $U \in \mathcal{U}^{\prod_{x \in X} Y_x}(g)$.
        Dann gilt:
        \begin{equation*}
        \begin{split}
        &\exists O \in \prod_{x \in X} \mathcal{T}_x : g \in O \subseteq U\\
        &\Rightarrow \exists x_1, \ldots, x_n \in X, 
        O_{x_1} \in \mathcal{T}_{x_1}, \ldots, O_{x_n} \in \mathcal{T}_{x_n} :
        g \in \bigcap_{k=1}^{n} \pi_{x_k}^{-1}(O_{x_k}) \subseteq O \subseteq U\\
        &\forall k \in \{1, \ldots, n \}: g \in \pi_{x_k}^{-1}(O_{x_k}) \in \prod_{x \in X} \mathcal{T}_x\\
        &\Rightarrow \forall k \in \{1, \ldots, n \} : \pi_{x_k}(g) \in O_{x_k} ( \text{Umgebung von} g(x_k)) \\
        &\overset{\text{VS}}{\Rightarrow} \text{ Wähle } i_k \in I :
        \forall i \succeq i_k : 
        \underbrace{(\pi_{x_k} \circ \varphi)(i) \in O_{x_k}}_{\Leftrightarrow \phi(i) \in \pi_{x_k}^{-1}(O_{x_k})}\\
        &\text{Wähle } i_k \in I : \forall i \succeq i_k : \forall k \in \{1, \ldots, n\} :
        \varphi(i) \in \pi_{x_k}^{-1}(O_{x_k})\\
        &\Leftrightarrow \varphi(i) \in \bigcap_{k=1}^{n} \pi_{x_k}^{-1}(O_{x_k}) \subseteq U\\
        \end{split}
        \end{equation*}
    \end{itemize}
    
\end{proof}

\mprop{}
{
    Sei $X$ überbzählbar, $\mid Y_k\mid \geq 2$ für alle $x \in X$.
    Dann gibt es keine Metrik $d$ auf $\prod_{x \in X} Y_x$ 
    so dass $d$ die Produkttopologie induziert.
}

\mlenma{}
{
    $\langle Z, d \rangle$ ein Metrischer Raum. $z \in Z \Rightarrow 
    \exists U_n, n \in \mathbb{N}$\footnote{ Abzählbar viele} 
    Umgebungen von z mit $\bigcap_{n=1}^{\infty} U_n = \{z\}$
}
\begin{proof}{Lemma: 5.2.5:}\\
    Sei $n \in \mathbb{N}$.
    Definiere $U_n := B_{1/2}(z)$.
    Sei:
    $$
    w \in \bigcap_{n=1}^{\infty} B_{1/2}(z) \Rightarrow d(w,z) < 1/n \quad 
    \forall n \in \mathbb{N}
    $$
    Dann gilt $d(w,z) = 0 \Rightarrow w = z$.
\end{proof}

\begin{myproof}{Proposition: 5.2.4:}\\
    Sei $f \in \prod_{x \in X} Y_x, \quad U_n \in \mathcal{U}^{\prod_{x \in X}Y_x}
    (f(x_n)) \forall n \in \mathbb{N}$.

    Wähle $  O_n \in \prod_{x \in X} \mathcal{T}_x : f \in O_n \subseteq U_n.$ \\
Vereinigung von endlichen Durchschnitten von Mengen aus (char. erzeugte Top.). \\
Wähle 
$$ 
\begin{cases}
x_{n,1}, \ldots, x_{n,m_n} \in X \\
O_{x_{n,1}}, \ldots, O_{x_{n,m_n}} \in \mathcal{T}_{x_{n,m_n}}
\end{cases}
$$
So dass: $f \in \bigcap_{k=1}^{m_n} \pi_{x_{n,k}}^{-1}(O_{x_{n,k}}) \subseteq O_n \subseteq U_n.$\\
$\Rightarrow f \in \bigcap_{n=1} \bigcap_{k=1}^{m_n} \pi_{x_{n,k}}^{-1}(O_{x_{n,k}}).$\\
Da $X$ überbzählbar ist, gibt es ein $x_0 \in \{x_{n,j} \mid n \in \mathbb{N},\, j \in \{1, \ldots, m_n\}\} \not\subseteq X$ \\
Das geht da X überbzählbar ist,
$Y_{x_0} \setminus \{f(x_0)\} \neq \emptyset
  \quad (\text{da } |Y_{x_0}| \ge 2 \ \forall x \in X) \\
  \rightsquigarrow \text{wähle } y \in Y_{x_0} \setminus \{f(x_0)\}$\\
Definiere:
$$
\tilde{f}(x) = 
  \begin{cases}
    f(x), & x \ne x_0, \\
    y, & x = x_0.
  \end{cases}
$$
$\Rightarrow \tilde{f} \neq f$ und\\
$\Rightarrow \tilde{f} \in \bigcap_{n\in \mathbb{N}}
\bigcap_{j=1}^{m_n} \pi_{x_{n,j}}^{-1}(O_{x_{n,j}}) \subseteq
\bigcap_{n \in \mathbb{N}} U_n.$\\
\end{myproof}

\nt
{
    \textbf{Wiederholung:} $\langle X, \mathcal{T} \rangle$ Topologischer Raum.\\
    $K \subseteq X$ heißt Kompakt 
    $\Leftrightarrow \forall C \in \mathcal{T} : \cup C \supseteq K 
    \Rightarrow (\exists C^{'} \subseteq C \text{ endlichen } : \cup C^{'} \supseteq K)$.
}

\mprop{}
{
    $\langle X, \mathcal{T} \rangle$ topologischer Raum.
    Dann sind folgende Aussagen äquivalent:
    \begin{itemize}
        \item[(i)] $K \subseteq X$ ist endlichen $\Rightarrow$ ist kompakt.
        \item[(ii)] $K_1, \dots, K_n \subseteq X$ sind kompakt $\Rightarrow$ 
        $K_1 \cup \ldots \cup K_n$ ist kompakt.

        \item[(iii)] $K \subseteq X$ ist kompakt $A \subseteq X$ abgeschlossen $\Rightarrow$
        $K \cap A$ ist kompakt.
        \item[(iv)] Sei $\langle Y, \mathcal{V} \rangle$ $K \subseteq X$ kompakt
        und $f : X \to Y$ stetig. Dann ist $f(K)$ kompakt.
    \end{itemize}
}

\begin{proof}{Proposition: 5.3.2:}\\
    \begin{itemize}
        \item[(i)]$K = \{x_1,\ldots,x_n\}$, $C \subseteq \mathcal{T}$ mit $\cup C \supseteq K$.\\
        Für $j \in \{1, \ldots, n\}$ wähle $O_j \in C$ mit $x_j \in O_j$.\\
        Dann ist $K = \{x_1, \ldots, x_n\} \subseteq \bigcup_{j=1}^{n} O_j$.
        \item[(ii)] Sei $C \subseteq \mathcal{T}$ mit $\cup C \supseteq K_1 \cup \ldots \cup K_n$.\\
        $\Rightarrow \forall j \in \{1, \ldots, n\} : \cup C \supseteq K_j$.\\
        $\Rightarrow \forall j \in \{1, \ldots, n\} : \exists C_j \subseteq C$ endlich
        mit $\cup C_j \supseteq K_j$.\\
        $\Rightarrow \bigcup \underbrace{C^{'} := \{O \mid \exists j \in \{1, \ldots, n\} : O \in C_j\}}_{\text{endlich}}$
        $\supseteq K_1 \cup \ldots \cup K_n$.
        \item[(iii)] Sei $C \subseteq \mathcal{T}$ mit $\cup C \supseteq K \cap A$.\\
        $C \cup \{X \setminus A\} \subseteq \mathcal{T} 
        \rightsquigarrow \bigcup C \cup (X \setminus A) \supseteq K \cap A$.\\
        Wähle $O_1, \ldots, O_m \in C$ so dass\\
        $O_1 \cup \ldots \cup O_n \cup (X \setminus A) \supseteq K$.\\
        Dann gilt $(O_1 \cup \ldots \cup O_n \cap (X \setminus A)) \cap A\supseteq K \cap A$.\\
        \item[(iv)] Sei $C \subseteq \mathcal{V}$ mit $\cup C = \bigcup_{O \in C} O \supseteq f(K)$.\\
        $\Rightarrow \underbrace{f^{-1}(\bigcup_{O \in C} O)}_{\bigcup_{O\in C}
        \underbrace{f^{-1}(O)}_{\text{offen}}} \supseteq K$.\\
        Definiere $D:= \{f^{-1}(O) \mid O \in C\} \subseteq \mathcal{T}$ und $\cup D \supseteq K$.\\
        Wähle $O_1, \ldots, O_n \in C$ so dass $\bigcup_{j=1}^{n} f^{-1}(O_j) \supseteq K$.\\
        $\Rightarrow f(\bigcup_{j=1}^{n} f^{-1}(O_j)) \supseteq f(K)$.\\
        wobei $f(\bigcup_{j=1}^{n} f^{-1}(O_j)) = \bigcup_{j=1}^{n} f(f^{-1}(O_j)) \subseteq \bigcup_{j=1}^{n} O_j$.\\
    \end{itemize}
\end{proof}


\begin{corollary}{}
    _
    \begin{itemize}
        \item[(i)] Sei $\langle X, \mathcal{T} \rangle$ Hausdorff'scher Raum: $K \subseteq X$ kompakt 
        $\Rightarrow K$ ist abgeschlossen.
        \item[(ii)] Sei $\langle X, \mathcal{T} \rangle$ Topologischer Raum, X kompakt,
        $\langle Y, \mathcal{V} \rangle$ Hausdorff'scher Raum,
        $f : X \to Y$ stetig und bijektiv. Dann ist $f^{-1} : Y \to X$ stetig.
    \end{itemize}
\end{corollary}

\mlenma{}{
    Sei $\langle X, \mathcal{T} \rangle$ ein Hausdorff'scher Raum.
    Und $K_1, K_2 \subseteq X$ kompakt, $\neq \emptyset$  und disjunkt.
    $\Rightarrow \exists O_1, O_2$ offen und disjunkt,
    so dass $K_1 \subseteq O_1$ und $K_2 \subseteq O_2$.
}

\begin{proof}{Lemma 5.3.4:}\\
  \begin{itemize}
    \item[1. Schritt:] Sei $x \in X, K \subseteq X$ kompakt, $x \notin K$.\\
    $\forall y \in K : x \neq y \overset{X \text{ Hausdorff}}{\Rightarrow} 
    \text{ wähle } O_y, \tilde{O}_y \text{ offen mit } 
    O_y \cap \tilde{O}_y = \emptyset, x \in O_y, y \in \tilde{O}_y.$\\
    $K = \bigcup_{y \in K} \{y\} \subseteq\tilde{O}_y 
    \overset{K \text{ kompakt}}{\Rightarrow}$
    wähle $y_1, \ldots, y_n \in K : K \subseteq \bigcup_{j=1}^{n} \tilde{O}_{y_j}.$\\
    $$
    \underbrace{\bigcup_{j=1}^n \tilde{O}_{y_j}}_{\text{offen, } \supseteq K}
    \cap
    \underbrace{\bigcap_{j=1}^n O_{y_j}}_{\text{offen, } x \in}
    = 
    \bigcup_{j=1}^n [\tilde{O}_{y_j} \cap \bigcap_{k=1}^n O_{y_k}]
    \subseteq
    \bigcup_{j=1}^n [\tilde{O}_{y_j} \cap O_{y_j}]
    = \emptyset
    $$
    \item[2. Schritt:] Sei $K_1, K_2 \subseteq X$ kompakt, disjunkt.\\
    Für jedes $x \in K_1$ wähle $O_x, \tilde{O}_x$ disjunkt und 
    $x \in \tilde{O}_x$, $K_2 \subseteq O_x$.\\
    $K_1 = \bigcup_{x \in K_1} \{x\} \subseteq \tilde{O}_x
    \overset{K_1 \text{ kompakt}}{\Rightarrow}$
    wähle $x_1, \ldots, x_m \in K_1 : K_1 \subseteq 
    \bigcup_{i=1}^{m} \tilde{O}_{x_i}.$\\
    $$
    \underbrace{\bigcup_{j=1}^n \tilde{O}_{x_j}}_{\text{offen, } \supseteq K_1}
    \cap
    \underbrace{\bigcap_{j=1}^n O_{x_j}}_{\text{offen, } \supseteq K_2}
    = \emptyset 
    \quad (\text{wie oben})
    $$
  \end{itemize}
\end{proof}



\begin{proof}{Korollar 5.3.3:}\\
    \begin{itemize}
        \item[(i)] Sei $K \subseteq X$ kompakt und $x \in X \setminus K$.\\
        $\overset{Lemma \ 5.3.4}{\Rightarrow} 
        \exists O_1, O_2 $ Offen und disjunkt, $x \in O_1, O_2 \supseteq K$\\

        $\Rightarrow x \in O_1 \subseteq X \setminus K$\\ 
        Also ist $X \setminus K$ offen und $K$ abgeschlossen.
        \item[(ii)] Betrachte $Y \overset{f^{-1}}{\longrightarrow}X$
        Sei $A \subseteq X$ abgeschlossen.\\
        $\overset{\text{ Proposition \ 5.3.2}}{\Rightarrow} A = A \cap X$ ist kompakt.\\
        $\Rightarrow f(A) \subseteq Y $ ist kompakt.\\
        $\overset{Y \text{ Hausdorff}}{\Rightarrow} f(A)$ ist abgeschlossen.\\
    \end{itemize}

\end{proof}