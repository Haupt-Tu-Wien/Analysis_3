\chapter{}
In der letzten haben wir begonnen Satzz 7.0.4 zu beweisen. 
Uns fehlt noch die Richtung (ii) $\Rightarrow$ (i).
Um die zu zeigen benötigen wir noch ein Lemma.
\mlenma{}
{
    Seien $(V_n)_{n \in \mathbb{N}}$ endliche und nicht leere Mengen in $X$. 
    Und sei 
    $$
    f: \dot{\bigcup_{n \geq 1}} V_n \to \dot{\bigcup_{n \geq 0}} V_n
    $$ eine Abbildung. 
    Wobei $\forall n \geq 1: f(V_n) \subseteq f(V_{n})\subseteq V_{n+1}$. \\
    Dann gibt es eine  Folge $(x_n)_{n \in \mathbb{N}}$ mit
    $$
    \forall n \in  \mathbb{N}: x_n \in V_n 
    \text{ und } \forall n \geq 1: f(x_n) = x_{n+1}.
    $$  
}

\begin{proof}{Lemma 8.0.1:}\\
    Sei 
    $$
    M:=\{(x_0,\dots,x_n)\mid N \in \mathbb{N}, \quad (\forall n \in \{1,\dots, N\}: x_n \in V_n )
    \text{ und } (\forall n \in \{1,\dots, N\}: f(x_n) = x_{n+1})\}
    $$
    Sei weiters $x_n \in V_n$ beliebig. \\
    Dann gilt $(f^{(N)}(x_N),\dots f \circ f (x_N), f(x_N), x_N) \in M$.
    $$
    \text{Definiere } g: 
    \begin{cases}
    \dot{\bigcup}_{n\geq 0} V_n \to M \\
    x \mapsto (f^{(N)}(x),\dots, f \circ f (x), f(x), x) \text{ Wähle N so dass für } x \in V_N
    \end{cases}
    $$
    Da $g$ injektiv ist, ist $M$ unendlich.\\
    Wähle  $z_0 \in V_0, \dots, z_n \in V_n$ so dass 
    $M_{z_0,\dots,z_n} := \{(x_0,\dots,x_k) \in M \mid N \geq n \text{ und } 
    \forall k \in \{0,\dots,n\}: x_k = z_k\}\}$\\
    Definiere die Rekursive Folge $(x_n)_{n \in \mathbb{N}}$ 
    so dass $\forall n \in \mathbb{N}: z_n \in V_n \text{ und } M_{z_0,\dots,z_n}$ ist unendlich.
    \begin{itemize}
        \item[n=0:] $V_n$ endlich $M$ unendlich ist. 
        $M=\bigcup_{x \in V_0} M_{x_0}\Rightarrow \exists z_0 \in V_0: M_{z_0}$ unendlich.
        \item[n$\to$ n+1:] Angenommen wir haben $z_0,\dots,z_n$ konstruiert.\\
        \begin{equation*}
        \begin{split}
        &\underbrace{M_{z_0,\dots,z_{n+1}}}_{\text{unendlich}} 
        = \bigcup_{x \in V_{n+1}} M_{z_0,\dots,z_n,x}\\
        &\Rightarrow \exists z_{n+1} \in V_{n+1}: M_{z_0,\dots,z_{n+1}}
        \end{split}
        \end{equation*}
        unendlich.\\
        $\forall n \in \mathbb{N} \exists (x_0,\dots,x_n) \in M_{z_0,\dots,z_n}$
        Damit gilt per Konstruktion: $(x_0,\dots,x_N) \in M, N \geq n, x_0 = z_0,\dots,x_n = z_n$
        für alle $k \in \{1,\dots,N\}:  f(x_k) = x_{k+1} 
        \Rightarrow f\underbrace{(x_n)}_{z_n} = \underbrace{x_{n+1}}_{z_{n+1}}$.
    \end{itemize}
\end{proof}

\nt{
    \textbf{Satz 7.0.4:}\\
    \begin{itemize}
    \item[(i)] $K$ ist kompakt in $\langle X, \mathcal{T}_d \rangle$.
    Wobei $\mathcal{T}_d$ die von $d$ induzierte Topologie ist.
    \item[(ii)] Jede Folge in $K$ hat eine gegen einen Punkt in $K$ konvergente Teilfolge.
    \item[(iii)] $K$ ist total beschränkt, und $\langle K, d\vert_{K \times K} \rangle$ 
    ist ein vollständig metrischer Raum.
\end{itemize}
    
}

\begin{proof}{(iii)$\Rightarrow$(i) Satz 7.0.5:}\\
\begin{itemize}
    \item[(1)] Wir konstruieren rekursiv eine Folge $(Q_n)^{\infty}_{n=1}$ von Funktionen
    mit 
    \begin{itemize}
        \item[(i)]$\forall n \in \mathbb{N}, n\geq 1; \forall A \in Q_n: \delta(A) \leq \frac{1}{n}$ 
        (wobei $\delta(A)$ der Durchmesser von $A$ ist).
        \item[(ii)] $\forall n \geq 1 \forall A \in Q_{n+1} \exists !B \in Q_n: A \subseteq B$
    \end{itemize}
    \begin{itemize}
        \item[n=1:] Wähle $M_1,\dots,M_n$ so dass
        $K \subseteq \bigcup_{i = 1}^{n} M_i$ und $\forall i \delta(M_i) \leq 1$.
        Wir definieren $A_m := (M_m \cap K) \setminus \bigcup_{i=1}^{m-1} (M_i \cap K)$.
        \begin{equation*}
        \begin{split}
            &\Rightarrow \bigcup_{m=1}^{n} A_m \\
            = \bigcup_{m=1}^{n} (M_m \cap K) 
            &= \left( \bigcup_{m=1}^{n} M_m \right) \cap K = K\\
        \end{split}
        \end{equation*}
        $\forall m \neq m' : \quad A_m \cap A_{m'} = \emptyset$\\
        Setze $Q_1 := \{A_1,\dots,A_n\}$.
        \item[n$\to$ n+1:] Angenommen wir haben $Q_1,\dots,Q_n$ konstruiert.\\
        Wähle $M_1,\dots,M_k$ so dass
        $K \subseteq \bigcup_{i = 1}^{k} M_i$ und 
        \(\forall i:\ \delta(M_i) \le \frac{1}{n+1}.\)
        Setze $$A_m := (M_m \cap K) \setminus \bigcup_{i=1}^{m-1} (M_i \cap K).$$
        Dann gilt wieder:
        $$
        \{A_1,\dots,A_N\}
        $$
        Sind Partitionen von K
    \end{itemize}
    Definiere $Q_{n+1} := \{ B \cap A_i \mid B \in Q_n,\ i = 1, \ldots, N \}$.Partition von K
\nt{
    \textbf{Begründung: Paarweise disjunkt und Vereinigung ergibt K}

\begin{itemize}
    \item \textbf{Paarweise disjunkt:}
    $$
        B \cap A_i,\quad B' \cap A_j
    $$
    sind disjunkt, falls entweder $i \neq j$ (da $A_i \cap A_j = \emptyset$) 
    oder $B \neq B'$ (da $B \cap B' = \emptyset$).
    
    \item \textbf{Vereinigung ergibt $K$:}
    \begin{equation*}
    \begin{split}
        &\bigcup_{B \in Q_n} \ \bigcup_{i=1}^N (B \cap A_i) 
        = \bigcup_{i=1}^N \ \bigcup_{B \in Q_n} (B \cap A_i) \\
        &= \bigcup_{i=1}^N ( A_i \cap \underbrace{\bigcup_{B \in Q_n} B}_{=K} ) 
        = \bigcup_{i=1}^N A_i = K.
    \end{split}
    \end{equation*}
\end{itemize}
}
Definiere $f: \dot{\bigcup}_{n\geq 2} \to \dot{\bigcup}_{n \geq 1}Q_n$
Sei $A \in Q_n$, Sei $n\geq 2$. So dass $A \in Q_n$ Sei $B \in Q_{n-1}$ mit $B \subseteq A$. 
Dann ist $f(A) := B$.
\item[(2)] Sei 
\end{itemize}

\end{proof}

\dfn{Relativ Kompakt}
{
    Eine Teilmenge $A$ eines topologischen Raumes heißt \textbf{relativ kompakt},
    wenn der Abschluss $\overline{A}$ von $A$ kompakt ist.
    Wir definieren $A_m := (M_m \cap K) \setminus \bigcup_{i=1}^{m-1} (M_i \cap K)$.
}