\chapter{}
In der letzten haben wir begonnen Satzz 7.0.4 zu beweisen. 
Uns fehlt noch die Richtung (ii) $\Rightarrow$ (i).
Um die zu zeigen benötigen wir noch ein Lemma.
\mlenma{Lemma von König}
{
    Seien $(V_n)_{n \in \mathbb{N}}$ endliche und nicht leere Mengen in $X$. 
    Und sei 
    $$
    f: \dot{\bigcup_{n \geq 1}} V_n \to \dot{\bigcup_{n \geq 0}} V_n
    $$ eine Abbildung. 
    Wobei $\forall n \geq 1: f(V_n) \subseteq f(V_{n})\subseteq V_{n+1}$. \\
    Dann gibt es eine  Folge $(x_n)_{n \in \mathbb{N}}$ mit
    $$
    \forall n \in  \mathbb{N}: x_n \in V_n 
    \text{ und } \forall n \geq 1: f(x_n) = x_{n+1}.
    $$  
}

\begin{proof}{Lemma 8.0.1:}\\
    Sei 
    $$
    M:=\{(x_0,\dots,x_n)\mid N \in \mathbb{N}, \quad (\forall n \in \{1,\dots, N\}: x_n \in V_n )
    \text{ und } (\forall n \in \{1,\dots, N\}: f(x_n) = x_{n+1})\}
    $$
    Sei weiters $x_n \in V_n$ beliebig. \\
    Dann gilt $(f^{(N)}(x_N),\dots f \circ f (x_N), f(x_N), x_N) \in M$.
    $$
    \text{Definiere } g: 
    \begin{cases}
    \dot{\bigcup}_{n\geq 0} V_n \to M \\
    x \mapsto (f^{(N)}(x),\dots, f \circ f (x), f(x), x) \text{ Wähle N so dass für } x \in V_N
    \end{cases}
    $$
    Da $g$ injektiv ist, ist $M$ unendlich.\\
    Wähle  $z_0 \in V_0, \dots, z_n \in V_n$ so dass 
    $M_{z_0,\dots,z_n} := \{(x_0,\dots,x_k) \in M \mid N \geq n \text{ und } 
    \forall k \in \{0,\dots,n\}: x_k = z_k\}\}$\\
    Definiere die Rekursive Folge $(x_n)_{n \in \mathbb{N}}$ 
    so dass $\forall n \in \mathbb{N}: z_n \in V_n \text{ und } M_{z_0,\dots,z_n}$ ist unendlich.
    \begin{itemize}
        \item[n=0:] $V_n$ endlich $M$ unendlich ist. 
        $M=\bigcup_{x \in V_0} M_{x_0}\Rightarrow \exists z_0 \in V_0: M_{z_0}$ unendlich.
        \item[n$\to$ n+1:] Angenommen wir haben $z_0,\dots,z_n$ konstruiert.\\
        \begin{equation*}
        \begin{split}
        &\underbrace{M_{z_0,\dots,z_{n+1}}}_{\text{unendlich}} 
        = \bigcup_{x \in V_{n+1}} M_{z_0,\dots,z_n,x}\\
        &\Rightarrow \exists z_{n+1} \in V_{n+1}: M_{z_0,\dots,z_{n+1}}
        \end{split}
        \end{equation*}
        unendlich.\\
        $\forall n \in \mathbb{N} \exists (x_0,\dots,x_n) \in M_{z_0,\dots,z_n}$
        Damit gilt per Konstruktion: $(x_0,\dots,x_N) \in M, N \geq n, x_0 = z_0,\dots,x_n = z_n$
        für alle $k \in \{1,\dots,N\}:  f(x_k) = x_{k+1} 
        \Rightarrow f\underbrace{(x_n)}_{z_n} = \underbrace{x_{n+1}}_{z_{n+1}}$.
    \end{itemize}
\end{proof}

\nt{
    \textbf{Satz 7.0.4:}\\
    \begin{itemize}
    \item[(i)] $K$ ist kompakt in $\langle X, \mathcal{T}_d \rangle$.
    Wobei $\mathcal{T}_d$ die von $d$ induzierte Topologie ist.
    \item[(ii)] Jede Folge in $K$ hat eine gegen einen Punkt in $K$ konvergente Teilfolge.
    \item[(iii)] $K$ ist total beschränkt, und $\langle K, d\vert_{K \times K} \rangle$ 
    ist ein vollständig metrischer Raum.
\end{itemize}
    
}

\begin{proof}{(iii)$\Rightarrow$(i) Satz 7.0.5:}\\
\begin{itemize}
    \item[Behauptung (1):]
     Wir konstruieren rekursiv eine Folge $(Q_n)^{\infty}_{n=1}$ von Funktionen
    mit 
    \begin{itemize}
        \item[(i)]$\forall n \in \mathbb{N}, n\geq 1; \forall A \in Q_n: \delta(A) \leq \frac{1}{n}$ 
        (wobei $\delta(A)$ der Durchmesser von $A$ ist).
        \item[(ii)] $\forall n \geq 1 \forall A \in Q_{n+1} \exists !B \in Q_n: A \subseteq B$
    \end{itemize}


    \textbf{Beweis Behauptung (1):}
    \begin{itemize}
        \item[n=1:] Wähle $M_1,\dots,M_n$ so dass
        $K \subseteq \bigcup_{i = 1}^{n} M_i$ und $\forall i \delta(M_i) \leq 1$.\\
        Wir definieren $A_m := (M_m \cap K) \setminus \bigcup_{i=1}^{m-1} (M_i \cap K)$.
        \begin{equation*}
        \begin{split}
            &\Rightarrow \bigcup_{m=1}^{n} A_m \\
            = \bigcup_{m=1}^{n} (M_m \cap K) 
            &= \left( \bigcup_{m=1}^{n} M_m \right) \cap K = K\\
        \end{split}
        \end{equation*}
        $\forall m \neq m' : \quad A_m \cap A_{m'} = \emptyset$\\
        Setze $Q_1 := \{A_1,\dots,A_n\}$.
        \item[n$\to$ n+1:] Angenommen wir haben $Q_1,\dots,Q_n$ konstruiert.\\
        Wähle $M_1,\dots,M_k$ so dass
        $K \subseteq \bigcup_{i = 1}^{k} M_i$ und 
        \(\forall i:\ \delta(M_i) \le \frac{1}{n+1}.\)
        Setze $$A_m := (M_m \cap K) \setminus \bigcup_{i=1}^{m-1} (M_i \cap K).$$
        Dann gilt wieder:
        $$
        \{A_1,\dots,A_N\}
        $$
        Sind Partitionen von K
    \end{itemize}
    Definiere $Q_{n+1} := \{ B \cap A_i \mid B \in Q_n,\ i = 1, \ldots, N \}$.Partition von K
\nt{
    \textbf{Begründung: Paarweise disjunkt und Vereinigung ergibt K}

\begin{itemize}
    \item \textbf{Paarweise disjunkt:}
    $$
        B \cap A_i,\quad B' \cap A_j
    $$
    sind disjunkt, falls entweder $i \neq j$ (da $A_i \cap A_j = \emptyset$) 
    oder $B \neq B'$ (da $B \cap B' = \emptyset$).
    
    \item \textbf{Vereinigung ergibt $K$:}
    \begin{equation*}
    \begin{split}
        &\bigcup_{B \in Q_n} \ \bigcup_{i=1}^N (B \cap A_i) 
        = \bigcup_{i=1}^N \ \bigcup_{B \in Q_n} (B \cap A_i) \\
        &= \bigcup_{i=1}^N ( A_i \cap \underbrace{\bigcup_{B \in Q_n} B}_{=K} ) 
        = \bigcup_{i=1}^N A_i = K.
    \end{split}
    \end{equation*}
\end{itemize}
}
Definiere 
$$
f: \dot{\bigcup}_{n\geq 2} \to \dot{\bigcup}_{n \geq 1}Q_n
$$
Sei $A \in Q_n$, Sei $n\geq 2$. So dass $A \in Q_n$ und $B \in Q_{n-1}$
 mit $B \subseteq A$. 
Damit $f(A) := B$.
\item[Behauptung(2):] Sei $(A_n)_{n}^{\infty}$ eine Folge mit $A_n \in Q_n \forall n$
Dann ($\forall n \geq 2$ gilt: $f(A_n) = A_{n-1}$.)
$\Rightarrow$
 ($\exists  x \in X \forall U \supset \mathcal{U}(x)
\exists n_0 \in \mathbb{N}\forall n \geq n_0: A_n \subseteq U$).\\


\textbf{Beweis Behauptung (2):}

$\forall n, m \in \mathbb{N}: n \geq m : A_m \subseteq A_n $
Wähle $x_n \in A_n$ und $x_m \in A_m$ so dass
$ d(x_m,x_n) \leq \delta(A_n)\leq \frac{1}{n}$
Also ist $(x_n)_{n \in \mathbb{N}}$ eine Cauchyfolge in $K$.\\
$$
\Rightarrow \exists x \in K : \lim_{n \to \infty} x_n = x.
$$
Sei $U \in \mathcal{U}(x)$ offen.
$\exists r > 0: U_r(x) \subseteq U$.
Wähle $n_0 \in \mathbb{N}$ so dass $\frac{1}{n_0} < \frac{r}{2}$.
Wähle weiters $n_2 \in \mathbb{N}$ so dass $\forall n \geq n_2: d(x_n,x) < \frac{r}{2}$.
Sei $n_0 \geq \max\{n_1,n_2\}$.
Betrachte nun $n \geq n_0$.
Damit gilt für $y \in A_n \subseteq A_{n_0}$ dass
$$
x_{n_0} \in A_{n_0} \Rightarrow d(y,x_{n_0}) \leq \delta(A_{n_0}) 
\leq \frac{1}{n_0} < \frac{r}{2}.
$$
Dann gilt für alle $n \geq n_0: d(x_n,x_{n_0}) < \frac{r}{2}$.
$\Rightarrow d(x,y) < r$. das heißt $y \in U_r(x) \subseteq U$.

\item[Behauptung (3):] 
Jede offene Überdeckung von $K$ besitzt eine endliche Teilüberdeckung.
Sei $C \in \mathcal{T}$ mit $\bigcup C \supseteq K$.
Wir definieren die Menge:
$$
\hat{Q}_n := \{ A \in Q_n \mid \forall O \in C: A \not\subseteq O \}.
$$


\textbf{Beweis Behauptung (3):}

\begin{itemize}
    \item[Fall 1:] $\exists n: \hat{Q}_n = \emptyset$\\
    $\forall A \in Q_n: \exists O \in C: A \subseteq O$.
    Sei $A \in Q_n$. Dann gibt es ein $O_A \in C$ mit $A \subseteq O_A$.
    $\Rightarrow \bigcup_{A \in Q_n} O_A \supseteq \bigcup_{A \in Q_n} A = K$.
    $Q_n$ ist endlich\\
    $\Rightarrow \{O_A \mid A \in Q_n\}$ ist eine endliche Teilüberdeckung von $C$.
    \item[Fall 2:] $\forall n: \hat{Q}_n \neq \emptyset$\\
    Alle $\hat{Q}_n$ sind nicht leer und endlich.\\
    Betrachte die funktion 
    $$
    \hat{f}:= f\vert_{\dot{\bigcup}_{n\geq 2}\hat{Q}_n} : 
    \dot{\bigcup}_{n\geq 2}\hat{Q}_n \to \dot{\bigcup}_{n \geq 1}\hat{Q}_n
    $$
    Wir wollen zeigen, dass die Zielmenge von $\hat{f}$
    die Menge  $\hat{Q}_{n\geq 1}$ ist.
    Sei $A \in \hat{Q}_n, f(A) \in Q_{n-1}$ mit $f(A) \supseteq A$.
    Angenommen $O \in C$ mit $f(A) \subseteq O$.
    Dann gilt auch $A \subseteq O \Rightarrow f(A) \in \hat{Q}_{n-1}$.\\
    Also 
    $$
    \hat{f}:
    \dot{\bigcup}_{n\geq 2}\hat{Q}_n \to \dot{\bigcup}_{n \geq 1}\hat{Q}_n
    $$.
    Wähle $(A_n)_{n=1}^{\infty}$ mit $A_n \in \hat{Q}_n$ und 
    $\forall n \geq 2: f(A_n) = A_{n-1}$.\\
    $\boxed{\textbf{Anwendung vom Lemma 8.0.1 (Lemma von König)}}$\\
    Wähle $x \in X$ so dass
    $$
    \forall U \in \mathcal{U}(x) \exists n_0 \in \mathbb{N} \forall n \geq n_0: A_n \subseteq U.
    $$
    Wähle $O\in C: x \in O \Rightarrow O \in \mathcal{U}(x)$ 
    und $n_0$ so dass $\forall n \geq n_0: A_n \subseteq O$.\\
    Widerspruch zur Definition von $\hat{Q}_n$.\\
\end{itemize}
    Also muss Fall 1 eintreten.

\end{itemize}

\end{proof}

\dfn{Relativ Kompakt}
{
    Eine Teilmenge $A$ eines topologischen Raumes heißt \textbf{relativ kompakt},
    wenn der Abschluss $\overline{A}$ von $A$ kompakt ist.
    Wir definieren $A_m := (M_m \cap K) \setminus \bigcup_{i=1}^{m-1} (M_i \cap K)$.
}

\begin{corollary}{}
    Sei $\langle X, d \rangle$ ein metrischer Raum. 
    $A \subseteq X$ ist relativ kompakt
    $\Leftrightarrow$
    $A$ ist total beschränkt
\end{corollary}

\begin{proof}{Korollar 7.0.6:}\\
    \begin{itemize}
        \item[(“$\Rightarrow$”)] Sei $A$ relativ kompakt. 
        Das heißt $\overline{A}$ kompakt. 
        \item[("$\Leftarrow$" )] Sei $A$ total beschränkt.
        $\Rightarrow \overline{A}$ ist total beschränkt.\\
        \nt{
        $$\delta(M) = sup\{d(x,y) \mid x,y \in M\} 
        = sup\{d(x,y) \mid x,y \in \overline{A}\} = \delta(\overline{A})
        $$
        }
        Sei $\epsilon > 0$. Wähle $M_1,\dots M_n$ 
        mit $\bigcup_{i=1}^n M_i \supseteq A$ und
        $\forall i: \delta(M_i) \geq \epsilon$ für $i \neq j$.
        $\Rightarrow \bigcup_{i=1}^n \overline{M_i} 
        = \overline{\bigcup_{i=1}^n M_i} \supseteq \overline{A}$.
        und $\forall i : \delta(M_i) \geq \epsilon$.
        Da $X$ vollständig und $\overline{A}$ abgeschlossen ist, ist
        $\overline{A}$ vollständig womit $\overline{A}$ kompakt ist.
    \end{itemize}
    
\end{proof}
\fbox{%
	\parbox{0.9\linewidth}{%
\thm{Achtung Fehlerhaft!}
{
    Sei $\langle X, \| \cdot \|_{\infty} \rangle$ ein vollständig normierter Raum.

}
\begin{proof}{Satz 8.0.4:}\\
    $\boxed{\text{Klassisches Argument für Vollständigkeit von } C(X,\mathbb{R})}:$\\
    Sei $(f_n)_{n \in \mathbb{N}}$ eine Cauchy-Folge in $(C(X,\mathbb{R}),
     \|\cdot\|_\infty)$.

    \begin{itemize}
        \item[(1)] Für jedes $x \in X$ gilt:
        $$
        \| f_n - f_m \|_\infty \to 0 \quad \Rightarrow \quad |f_n(x) - f_m(x)| \to 0.
        $$
        Also ist $(f_n(x))_{n \in \mathbb{N}}$ eine Cauchy-Folge in $\mathbb{R}$.

        \item[(2)] Definiere $f : X \to \mathbb{R}$ durch
        $$
        f(x) := \lim_{n \to \infty} f_n(x).
        $$

        \item[(3)] Sei $\varepsilon > 0$. 
        Da $(f_n)$ eine Cauchy-Folge in $C(X,\mathbb{R})$ ist, existiert $n_0 \in \mathbb{N}$, sodass
        $$
        \forall n,m \geq n_0: \ \| f_n - f_m \|_\infty \le \varepsilon.
        $$
        Für alle $x \in X$ gilt dann
        $$
        |f_n(x) - f_m(x)| \le \varepsilon.
        $$
        Lässt man $m \to \infty$ gehen, folgt
        $$
        |f_n(x) - f(x)| \le \varepsilon,
        $$
        da der Betrag stetig ist und der Grenzübergang punktweise konvergiert.
    \end{itemize}

    Also gilt:
    $$
    \forall x \in X,\ \forall n \ge n_0:\ |f(x) - f_n(x)| \le \varepsilon,
    $$
    und somit
    $$
    \forall n \ge n_0:\ \sup_{x \in X} |f(x) - f_n(x)| \le \varepsilon.
    $$
    Das heißt:
    $$
    \| f - f_n \|_\infty \le \varepsilon \quad \text{für alle } n \ge n_0.
    $$

    Damit konvergiert $(f_n)$ gleichmäßig gegen $f$, und $f \in C(X,\mathbb{R})$, 
    da der gleichmäßige Grenzwert stetiger Funktionen wieder stetig ist. 
    Somit ist $(C(X,\mathbb{R}), \|\cdot\|_\infty)$ vollständig.
\end{proof}
}%
}

\thm{Satz von Asela Ascoli}
{
    Sei $\langle X, \mathcal{T} \rangle$ ein kompakter Topologischer Raum.
    Eine Teilmenge $\mathcal{G} \subseteq C_b(X, \mathbb{R})$ bzw. 
    $\mathcal{G} \subseteq C_b(X, \mathbb{C})$ ist genau dann bezüglich 
    der von $\|\cdot\|_\infty$ auf $C_b(X, \mathbb{R})$ bzw. 
    $C_b(X, \mathbb{C})$ induzierten Metrik total beschränkt, 
    wenn $\mathcal{G}$ punktweise beschränkt und gleichgradig stetig ist.

    Das heißt, es gelten äquivalent die folgenden beiden Eigenschaften:
    \begin{itemize}
        \item[(i)] Für jedes $x \in X$ ist die Menge 
        $$
        \{ f(x) \mid f \in \mathcal{G} \}
        $$
        beschränkt.

        \item[(ii)] Für jedes $x \in X$ und jedes $\varepsilon > 0$ 
        existiert eine Umgebung $V \in \mathcal{U}(x)$, so dass
        $$
        |f(y) - f(x)| < \varepsilon 
        \quad \text{für alle } y \in V \text{ und alle } f \in \mathcal{G}.
        $$
    \end{itemize}
}


\begin{proof}{Satz 8.0.4 (Ascoli):}\\
    Sei $\mathcal{G}$ wie oben definiert und erfüllt (i) und (ii). 
    Wir zeigen, dass $\mathcal{G}$ total beschränkt ist. 

    Sei $\varepsilon > 0$ gegeben. 
    $ \forall x \in X$ wähle eine Umgebung $V_x \in \mathcal{U}(x)$, 
    so dass $\forall y \in V_x$ und $\forall f \in \mathcal{G}$ gilt:
    $$
    |f(y) - f(x)| < \varepsilon.
    $$

    Da $\{V_x : x \in X\}$ eine offene Überdeckung von $X$ ist 
    und $X$ kompakt, existieren endlich viele Punkte 
    $x_1, \dots, x_n \in X$ mit
    $$
    X = \bigcup_{i=1}^{n} V_{x_i}.
    $$

    Für $i = 1, \dots, n$ gilt also
    $$
    |f(x) - f(x_i)| < \varepsilon 
    \quad \text{falls } x \in V_{x_i},\ f \in \mathcal{G}. 
    $$

    Nach (i) existiert ein $c > 0$ mit 
    $$
    c := \sup\{ |f(x_i)| : i = 1, \dots, n,\ f \in \mathcal{G} \} < +\infty.
    $$
    Betrachte die Abgeschlossene  Kugel $K := \overline{B_c^{\mathbb{R}^n}(0)}$ 
    bzw. $K := \overline{B_c^{\mathbb{C}^n}(0)}$ im $\mathbb{R}^n$ bzw. $\mathbb{C}^n$ 
    bezüglich der $\|\cdot\|_\infty$–Norm und definieren
    $$
    p : \mathcal{G} \to K, \quad p(f) := (f(x_1), \dots, f(x_n))^T.
    $$

    Da $K$ kompakt ist, folgt dass $K$
    total beschränkt ist. 
    Nach \textbf{Korollar 8.0.3} ist daher auch $p(\mathcal{G})$ 
    total beschränkt. 
    $\Rightarrow$ existieren endlich viele $f_1, \dots, f_m \in \mathcal{G}$, so dass 
    für jedes $f \in \mathcal{G}$ ein $k \in \{1, \dots, m\}$ mit
    $$
    \|p(f) - p(f_k)\|_\infty < \varepsilon
    $$
    existiert, d.\,h.
    $$
    |f(x_i) - f_k(x_i)| < \varepsilon \quad \text{für } i = 1, \dots, n.
    $$

    Wähle $f \in \mathcal{G}$ so das $|f(x_i) - f_k(x_i)| < \varepsilon \quad \text{für } i = 1, \dots, n.$
    
    Für jedes $x \in X$ liegt $x$ in einem $V_{x_i}$, und daher folgt von 
    Oben ($|f(x) - f(x_i)| < \varepsilon 
    \text{falls } x \in V_{x_i},\ f \in \mathcal{G}.$)
    

    \begin{equation*}
    \begin{split}
        |f(x) - f_k(x)| 
        &\le |f(x) - f(x_i)| + |f(x_i) - f_k(x_i)| + |f_k(x_i) - f_k(x)| \\
        &< \varepsilon + \varepsilon + \varepsilon = 3\varepsilon.
    \end{split}
    \end{equation*}
    Also gilt $\|f - f_k\|_\infty \le 3\varepsilon$, womit die offenen Kugeln 
    $B_{4\varepsilon}(f_k)$, $k = 1, \dots, m$, ganz $\mathcal{G}$ überdecken.

    \bigskip
    Zur Umkehrung: 
    Sei $\mathcal{G}$ total beschränkt. 
    Dann ist $\mathcal{G}$ beschränkt in $(C_b(X,\mathbb{R}),\|\cdot\|_\infty)$ 
    bzw. $(C_b(X,\mathbb{C}),\|\cdot\|_\infty)$ und somit punktweise beschränkt
    damit  haben wir getzeigt das (i) gilt.

    Wegen der totalen Beschränktheit existieren für jedes $\varepsilon > 0$ 
    endlich viele 
    $f_1, \dots, f_m \in \mathcal{G}$ mit 
    $\forall f \in \mathcal{G}\ \exists k:\ \|f - f_k\|_\infty < \varepsilon$.\\
    Sei $x \in X$. 
    Durch Stetigkeit jedes $f_k$ existieren Umgebungen $V_k \in \mathcal{U}(x)$ 
    mit $|f_k(y) - f_k(x)| < \varepsilon$ für alle $y \in V_k$.
    Setze $V := V_1 \cap \dots \cap V_m \in \mathcal{U}(x)$.
    Für $f \in \mathcal{G}$, $y \in V$ und passendes $k$ gilt dann:
    \begin{equation*}
    \begin{split}
        |f(y) - f(x)| 
        &\le |f(y) - f_k(y)| + |f_k(y) - f_k(x)| + |f_k(x) - f(x)| \\
        &< \varepsilon + \varepsilon + \varepsilon = 3\varepsilon.
    \end{split}
    \end{equation*}
    Damit ist auch (ii) erfüllt.
\end{proof}