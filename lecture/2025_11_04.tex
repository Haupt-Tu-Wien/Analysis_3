\chapter{}
\begin{corollary}{Um das Lemma anzuwenden}
Sei $\mathcal{A}\subseteq C(X,\mathbb{R})$ Punktetrennende Unteralgebra
mit $1 \in \mathcal{A}, A,B \subseteq X$ Abgeschliossen, $A \cap B = \emptyset$.
$$\epsilon > 0 \Rightarrow \exists f \in A: \forall x \in X:0\leq f(x)\leq 1
\forall x \in B: f(x) \geq 1-\epsilon$$
\end{corollary}

\begin{proof}
    $\mathcal{A}= \emptyset$.Dann ist das ganze trivial.
    Sei also $A \neq \emptyset$. Für $a \in A$ Wähle $V_a \in \mathcal{a}$ offen.\\
    Und $V_a \cap B = \emptyset$

    \nt{
        $x,y\in X,x\neq y$ Wähle $f \in \mathcal{A}:f(x)\neq f(y)$
        Dann:
        $$
        f^{-1}((f(x)-\delta,f(x)+\delta)), f^{-1}((f(y)-\delta,f(y)+\delta))
        $$
        Dann wähle $\delta := |\frac{f(x)-f(y)}{3}|$.\\
        Dann sind die beiden Mengen disjunkt. \\
        (Beweis $\langle X,d\rangle$ erfüllt (T2))
    }
    $A \subseteq \bigcup_{a \in A}V_a\Rightarrow$ Wähöe $a_1,...,a_n \in A:
    A \subseteq \bigcup_{i=1}^{n}V_{a_i}$
\end{proof}

\section{Korrektur}

In der Vorlesungvom 2025.11.04 gab es eine kleine große verwirrung.
Also ich habe nicht nur mein Tablet vergessen, es ist noch mehr schief gegeanen (What a day).
Aufjedenfall hat uns, unser Professor (The one and only Woracek) eine Korrektur zukommen lassen.
Hier ist sie:

\mlenma{Das Lemma (starke Variante)}
{
 $ \forall A \subseteq X$ abgeschlossen
$$\forall z \in X \setminus A
\exists V \subseteq X
\forall \epsilon > 0
\exists f \in \mathcal A$$

sodass

V ist offene Umgebung von z
$$\wedge
V \cap A = \emptyset
\wedge
\forall x \in X : 0 \leq f(x) \leq 1
\wedge
\forall x \in V : f(x) \leq \epsilon
\wedge
\forall x \in A : f(x) \geq 1-\epsilon$$
}

\begin{proof}{Lemma 10.1.1 (starke Variante)}\\
Wir gehen den beweis schritt für schritt durch
   \begin{itemize}
    \item[Schritt 1:] hier haben wir gezeigt dass

$\forall A \subseteq X$ abgeschlossen
$$
\forall z \in X \setminus A
\exists V \subseteq X
\exists h \in \mathcal A
\exists \delta \in (0,1]
$$

sodass

$V$ ist offene Umgebung von z
$$\wedge
V \cap A = \emptyset
\wedge
\forall x \in X : 0 \leq h(x) \leq 1
\wedge
\forall x \in V : h(x) \leq \delta/3
\wedge
\forall x \in A : h(x) \geq \delta$$

\item[Schritt 2:] hier haben wir gezeigt dass

Sei h wie in Schritt 1 und definiere 
$p_n := (1-h^n)^{k^n}$. Dann gilt $"\lim_{n\to\infty} p_n(x) = 0$
 gleichmaessig für $x \in A und \lim_{n\to\infty} p_n(x) = 1$ gleichmaessig für 
 $x \in V$".

Der Beweis des oben stehenden Lemmas (starke Variante) ist nun fertig, weil:

Seien A,z gegeben. Waehle V wie in Schritt 1
. Sei $\epsilon>0$gegeben. Waehle n so gross, dass 
$$p_n(x) \leq \epsilon fuer alle x \in A$$ und 
$$p_n(x) \geq 1-\epsilon \forall x \in V. Setze f := 1-p_n.$$
\end{itemize}
\end{proof}

zum Korollar:
\begin{corollary}{}
$$\forall A,B \subseteq X$
abgeschlossen und disjunkt
$$
\forall \epsilon > 0
\exists f \in \mathcal A
$$

sodass

$$
\forall x \in X : 0 \leq f(x) \leq 1
\wedge
\forall x \in A : f(x) \leq \epsilon
\wedge
\forall x \in B : f(x) \geq 1-\epsilon
$$
\end{corollary}

\begin{proof}{Korollar: 10.1.2}\\
    Seien $A,B,\epsilon$ gegeben
\begin{itemize}
\item  Fuer jedes $z \in A $wenden wir das Lemma an mit der Menge B und dem Punkt z. Dies gibt $V_z$.
\item Nun reichen endlich viele $V_z$ aus um A zu ueberdecken, waehle solche: $V_{z_1},\dots,V_{z_N}.$
\item Aus der Eigenschaft des Lemmas waehle nun $f_{z_j}$ für $\epsilon/N$
\item Betrachte das Produkt dieser $f_{z_j}$. Mit Bernoullischer Ungleichung sieht man dass das Gewuenschte gilt.
\end{itemize}
\end{proof}

\section{Weiter in der Vorlesung}
Nach diesem einschub kommen wir nun zum Nächsten Punkt (alias Satz 10.2.4)

\thm{}
{
    Sei $X$ Kompakt, $C(CX,\mathbb{R})$ mit Unteralgebra, nirgends verschwindend und 
    abgeschlossen bezüglich Komplexer konjugation.
    $\Rightarrow \overline{\mathcal{A}}^{\infty}=C(X,\mathbb{R})$
}

\begin{proof}{Satz; 10.2.1}\\
    Sei $B=\{f\in \mathcal{A}\mid f(x)\in \mathbb{R}\}\subseteq C(X,\mathbb{R})$
    Unteralgebra von $\mathcal{A}$.
    Es gelte $\forall \in \mathcal{A}: Re f, im f\in B: Re f = \frac{1}{2}(f+\overline{f})$\\
    $Im f = \frac{1}{2i}(f-\overline{f})$\\
    Sei $x \in X$ Wähle $f \in \mathcal{A}: f(x) \neq f(y) \Rightarrow$ entweder $Re f(x) \neq Re f(y)$
    oder $Im f(x) \neq Im f(y) \Rightarrow B$ ist punktetrennend.\\
    Aus dem Korollar folgt nun $\overline{B}^{\infty} = C(X,\mathbb{R}) \Rightarrow
    \overline{\mathcal{A}}^{\infty} = C(X,\mathbb{R})$
    Sei $f \in C(X,\mathbb{C})$ $Re f \in \overline{B}^{\infty} \subseteq \overline{\mathcal{A}}^{\infty} \Rightarrow
    f = Re f + i Im f \in \overline{\mathcal{A}}^{\infty}$
\end{proof}

\thm{}{
    $X$ kompakt, $T_2$.  $C(X,\mathbb{R}) \supseteq \mathcal{A}$ Unteralgebra $\Rightarrow \mathcal{A}$ ist:
    $$
        \mathcal{A}^{t^*} := \{ f \in C(X,\mathbb{R}) \,\mid, 
        \ker f \supseteq \bigcap_{g \in \mathcal{A}} \ker g \}.
    $$
}

\begin{proof}{10.2.2}
    \begin{itemize}
        \item[" $\subseteq$":] Sei $f \in \mathcal{A}^{t^*}$. \\
    Wähle $(g_n)_{n=1}^\infty$, $g_n \in \mathcal{A}$ mit $g_n \to f$.

    $\forall x,y$ mit $e(x) = e(y)$ gilt:\\
    $0 \in \ker g_n \quad\Rightarrow\quad g_n(x) = g_n(y)$  für alle $n$.\\
    $\Rightarrow g_n(x) = g_n(y) \to f(x) = f(y)$.
        \item[“$\supseteq$”] Betrachte 
    $$
        Y := X \,\big/\, \bigcap_{g \in \mathcal{A}} \ker g.
    $$
    Sei $q : X \to Y$ stetig.\\
    $\Rightarrow \ker q$ ist Algebra.

    $e(x_1) = e(x_2) \Rightarrow q(x_1) = q(x_2)$.
    \end{itemize}
\end{proof}