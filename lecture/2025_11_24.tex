\chapter{}
In der Vorlesung geht es um die Einführung der Initialen Topologien
-Ich denke das es darum geht anhand von dem was mein Auge erblickt-

\section{Kommutative Diagramme zur Veranschaulichung}
Weil die Konstrucktion bei einem normalen (Auch bei nicht normalen) menschen
ein paar knoten im kopf verursacht, wollen wir uns das ganze mal 
mit Bildern anschauen (Kommutative Diagramme).\\
Seien $X_i, i\in I$ Mengen "mit Topologien" und 
$
\Phi:
\begin{cases} Y\to \prod_{i\in I}X_i\\
    (\alpha_i(y))_{i\in I} \mapsto X_i
\end{cases}
$
eine Abbildung.\\
Wobei $\alpha_{i}: \pi_i \circ \Phi,  \forall i \in I$  mit $\pi_i$
die Projektion auf die $i$-te Komponente ist und damit ist $\pi_i$ stetig     
$\forall i \in I$.\\

\begin{tikzpicture}[>=stealth, xscale=1.2, yscale=1.2]

% X oben
\node (Y) at (0,0) {$Y$};

% Xi und Xj darunter
\node (Xi) at (0,-2) {$\prod_{i\in I}X_i$};

% Zielräume ganz unten (4 Stück)
\node (Xk-1) at (-4,-4) {$(X_{k-1},\mathcal{T}_{k-1})$,};
\node (dots) at (-3,-4) {\quad$\dots$,};
\node (Xk) at (-2,-4) {$(X_{k},\mathcal{T}_{k})$,};
\node (X0) at (0,-4) {$\dots$};
\node (Xj) at (2,-4) {,$(X_{j},\mathcal{T}_{j})$,};
\node (dots2) at (3,-4) {$\dots$,};
\node (Xj+1) at (4,-4) {\quad ,$(X_{j+1},\mathcal{T}_{j+1})$};

% f_i, f_j
\draw[->, dashed] (Y) -- node[left] {$\Phi$} (Xi);

% Xi → Ziele
\draw[->] (Xi) -- node[left] {$\pi_{k}$} (Xk);
\draw[->] (Xi) -- node[right] {$\pi_{j}$} (Xj);

% gestrichelte Kompositionen
\draw[->] (Y) .. controls (-2,-1) .. node[left] {$\alpha_{k}$} (Xk);
\draw[->] (Y) .. controls (2,-1) .. node[right] {$\alpha_{j}$} (Xj);

\end{tikzpicture}

Davon ausgehend wollen wir eine Topologie auf $Y$ definieren, die sogenannte \textbf{initiale Topologie} 
bezüglich der Familie von Abbildungen $(\alpha_i)_{i\in I}$.\\
 

\begin{tikzpicture}[>=stealth, xscale=1.2, yscale=1.2]

% X oben
\node (Y) at (0,0) {$\langle Y,\mathcal{V}\rangle$};

% Xi und Xj darunter
\node (Xi) at (0,-2) {$\langle \prod_{i\in I}X_i,\prod_{i\in I}\mathcal{T}_i \rangle$};

% Zielräume ganz unten (4 Stück)
\node (Xk-1) at (-4,-4) {$\langle X_{k-1},\mathcal{T}_{k-1}\rangle$,};
\node (dots) at (-3,-4) {\quad$\dots$,};
\node (Xk) at (-2,-4) {$\langle X_{k},\mathcal{T}_{k}\rangle$,};
\node (X0) at (0,-4) {$\dots$};
\node (Xj) at (2,-4) {,$\langle X_{j},\mathcal{T}_{j}\rangle$,};
\node (dots2) at (3,-4) {$\dots$,};
\node (Xj+1) at (4,-4) {\quad ,$\langle X_{j+1},\mathcal{T}_{j+1}\rangle$};

% f_i, f_j
\draw[->, dashed] (Y) -- node[left] {$\Phi$} (Xi);

% Xi → Ziele
\draw[->] (Xi) -- node[left] {$\pi_{k}$} (Xk);
\draw[->] (Xi) -- node[right] {$\pi_{j}$} (Xj);

% gestrichelte Kompositionen
\draw[->] (Y) .. controls (-2,-1) .. node[left] {$\alpha_{k}$} (Xk);
\draw[->] (Y) .. controls (2,-1) .. node[right] {$\alpha_{j}$} (Xj);

\end{tikzpicture}

\nt{
    $\prod_{i \in I} \mathcal{T}_i$ ist die Kleinste Topologie mit 
    $$\bigcup_{i \in I}\{\pi^{-1}(O_i)\mid \mathcal{O}_i \in \mathcal{T}_i\}$$
}
\section{Konstrucktion ohne Zeichnen}

Im folgenden wollen wir zeigen das wenn gilt: 
Seien $X_i, i \in I$ Mengen mit Topologien $\mathcal{T}_i$ und 
$\alpha_{i}: \pi_i \circ \Phi \forall i \in I$ und $\pi_i$ stetig $\forall i \in I$
Dann ist $\Phi$ stetig

\dfn{
    Seien $\rangle X_i,\mathcal{T}_i \langle, i \in I$ topologische, $i \in I$
    $X$ eine Menge und $f_i: X \to X_i$ Abbildungen, $\forall i \in I$.
    Die \textbf{initiale Topologie} $\mathcal{T_{\text{init}}}$ auf $X$ bezüglich der Familie von Abbildungen 
    $(f_i)_{i \in I}$ ist die kleinste Topologie auf $X$, so dass alle Abbildungen $f_i$ stetig sind.   
    $$
     \mathcal{T}_{\text{init}} := \text{Topologie erzeugt von } \bigcup_{i \in I}\{f_i^{-1}(O_i) \mid O_i \in \mathcal{T}_i\} 
    $$
}

\thm{}
{
    Seien $\langle X_i,\mathcal{T}_i \rangle,  i \in I$ topologische Räume, $X$ eine Menge und 
    $f_i: X \to X_i$ Abbildungen, $\forall i \in I$. 
    Dann ist die initiale Topologie $\mathcal{T}_{\text{init}}$ auf $X$ bezüglich der Familie von Abbildungen 
    $(f_i)_{i \in I}$ gegeben durch 
}